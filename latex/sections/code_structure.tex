\documentclass[../main.tex]{subfiles}

\begin{document}

\chapter{代码主体结构}
\vspace{-2cm}

简要总结

\section{整体框架}

- 参考 \url{https://github.com/XPixelGroup/BasicSR/blob/master/docs/DesignConvention_CN.md}

- 代码接口在 http://basicsr.readthedocs.io/,后续看看能否作为附录存在

%##################################################################################################
\begin{figure}[t]
    %\vspace{-0.5cm}
    \begin{center}
        %\fbox{\rule{0pt}{2.5in} \rule{0.9\linewidth}{0pt}}
        \includegraphics[width=\linewidth]{figures/basicsr_logo.png}
        \vspace{-1cm}
        \caption{图标题。BasicSR Logo。}
        \label{fig:logo}
    \end{center}
    %\vspace{-0.7cm}
\end{figure}
%##################################################################################################

\section{配置(Options)与注册器(Register)}

- 态实例化与REGISTER注册机制

- 约定

- 如何避免重复的类名和函数名

% \begin{minted}[xleftmargin=20pt,linenos,bgcolor=bg]{python}
% class RepConv(nn.Module):
%     """Re-parameterizable block for RepSR."""

%     def __init__(self,
%                  in_channels,
%                  out_channels,
%                  kernel_size,
%                  stride=1,
%                  padding=0,
%                  dilation=1,
%                  groups=1,
%                  padding_mode='zeros',
%                  deploy=False,
%                  width_multiplier=2,
%                  with_bn=True,
%                  frozen_bn=False):
%         super(RepConv, self).__init__()
%         self.deploy = deploy
%         self.in_channels = in_channels
%         self.out_channels = out_channels
%         self.kernel_size = kernel_size
%         self.stride = stride
%         self.padding = padding
%         self.dilation = dilation
%         self.groups = groups
%         self.with_bn = with_bn
%         self.frozen_bn = frozen_bn

%         self.mid_channels = out_channels * width_multiplier
%         self.rep_c1_2 = nn.Conv2d(
%             in_channels=self.mid_channels,
%             out_channels=out_channels,
%             kernel_size=1,
%             stride=stride,
%             padding=0,
%             groups=groups,
%             bias=True)

%         # initialization
%         init_list = [
%             self.rep_identity, self.rep_c3_1, self.rep_c1_1, self.rep_c3_2,
%             self.rep_c1_2
%         ]
%         if with_bn:
%             init_list.extend([self.rep_bn_1, self.rep_bn_2])
%         default_init_weights(init_list, scale=0.1)

%     def forward(self, inputs, frozen_bn=None):
%         if frozen_bn is None:
%             frozen_bn = self.frozen_bn

%         if hasattr(self, 'rep_merge'):
%             return self.rep_merge(inputs)
%         if self.with_bn:
%             id_out = self.rep_identity(inputs)
%             out_1 = self.rep_c1_1(
%                 self.rep_bn_1(self.rep_c3_1(inputs), frozen_bn))
%             out_2 = self.rep_c1_2(
%                 self.rep_bn_2(self.rep_c3_2(inputs), frozen_bn))
%         else:
%             id_out = self.rep_identity(inputs)
%             out_1 = self.rep_c1_1(self.rep_c3_1(inputs))
%             out_2 = self.rep_c1_2(self.rep_c3_2(inputs))

%         return id_out + out_1 + out_2
% \end{minted}


\section{数据(Data Loader)}

\section{网络结构(Architecture)}


本文档旨在完整地介绍 BasicSR 的设计和框架,为入门者提供一份上手指南,为使用者提供一份日常参考。

本文档不涉及具体函数和代码的介绍。如果需要具体函数和代码的介绍,请查阅 BasicSR 的在线 API 文档。我们更推荐读者直接查看代码,这样可以更加完整、细致的了解实现细节。

\begin{hl} % ---------------- Highlight block ---------------- %
	\textbf{BasicSR API 文档}

	BasicSR 的 API 文档是实时更新的,并且发布在 readthedocs.io 网站上:

	\url{https://basicsr.readthedocs.io/en/latest/}

	国内可能访问速度缓慢,后续我们会考虑导出 PDF 文档作为附录。
\end{hl}

\section{模型(Model)}

模型部分定义了许多模型级别的操作,比如训练设置、训练数据如何送入模型、优化器的选择、损失函数计算、模型参数更新、训练中验证、测试过程等。模型部分定义在:

\dirtree{%
	.1 BasicSR.
	.2 basicsr.
	.3 models.
}

目前BasicSR支持的模型有:

\begin{table}[h]
\begin{center}
\begin{tabular}{|l|l|l|}
\hline
\textbf{类}     & \textbf{描述}    & \textbf{支持的算法}                            \\ \hline
Base\_Model      & 抽象基类, 定义了共有的函数 &                                           \\ \hline
SR\_Model        & 基础图像超分类        &  \begin{tabular}[c]{@{}l@{}}SRCNN, EDSR, SRResNet, \\ RCAN, RRDBNet, etc\end{tabular} \\ \hline
SRGANModel     & SRGAN图像超分类     & SRGAN                                     \\ \hline
ESRGAN\_Model    & ESRGAN图像超分类    & ESRGAN                                    \\ \hline
VideoBase\_Model & 基础视频超分类        &                                           \\ \hline
EDVR\_Model      & EDVR视频超分类      & EDVR                                      \\ \hline
StyleGAN2\_Model & StyleGAN2图像生成类 & StyleGAN2                                 \\ \hline
\end{tabular}
\end{center}
\caption{支持的模型。}
      %
\end{table}


为增加模型间的复用, 我们大量使用了继承, 以下为各个模型之间的继承关系:
\newline
\dirtree{%
	.1 Base\_Model.
	.2 SR\_Model.
	.3 ESRGAN\_Model.
	.2 VideoBase\_Model.
	.3 EDVR\_Model.
	.2 StyleGAN2\_Model.
}


下面具体介绍一些重要模型及其功能,以及如何按照自己的需求自定义新的模型。

\subsection{Base Model}
Base Model是所有模型的基类,定义一些共同操作。比如:

输出网络信息:
\begin{minted}[xleftmargin=20pt,linenos,bgcolor=bg]{python}
print_network(self, net)
\end{minted}

保存模型、训练状态、加载模型:
\begin{minted}[xleftmargin=20pt,linenos,bgcolor=bg]{python}
save_network(self, net, net_label, current_iter, param_key='params')
save_training_state(self, epoch, current_iter)
load_network(self, net, load_path, strict=True, param_key='params')
\end{minted}

定义初始学习率、学习率更新:
\begin{minted}[xleftmargin=20pt,linenos,bgcolor=bg]{python}
_get_init_lr(self)
update_learning_rate(self, current_iter, warmup_iter=-1)
\end{minted}

\begin{hl} % ---------------- Highlight block ---------------- %
Base Model的很多函数,可以在继承后的模型中根据要求来重写,比如不同模型读取数据的方式不同,则会重写不同的 feed\_data() 函数。
\end{hl}

\subsection{其他模型,以SR Model为例}\label{Model:SR Model}

SR Model是图像超分辨率模型的类,定义了基础的单张图像超分辨率模型。下面从一个模型的训练角度展示其中重要的函数:

\begin{minted}[xleftmargin=20pt,linenos,bgcolor=bg]{python}
init_training_settings(self)
\end{minted}

初始化训练设置。包括优化器、损失函数的定义,学习率的初始化等。

\begin{minted}[xleftmargin=20pt,linenos,bgcolor=bg]{python}
feed_data(self, data)
\end{minted}

把训练数据送入模型。这里是从dataloader中取出数据,用于训练或测试。在SR Model中,每次取用一个batch(n,c,h,w)数量的LR和GT图像。

其他模型对batch做不同操作时,经常会改写这个函数。比如只读取GT、读取额外label、读取图像路径、对读取的数据增加degradation等操作,都通过修改feed\_data来实现。


\begin{minted}[xleftmargin=20pt,linenos,bgcolor=bg]{python}
optimize_parameters(self, current_iter)
\end{minted}

更新模型参数。在这个函数中,会实际调用模型跑出SR图像,再将SR与GT代入损失函数计算loss,并且进行loss的回传和参数的更新。当有多个loss或需要自己添加loss时,或者需要特殊的更新步骤时,修改这里。

比如在GAN系列的Model中,这个函数就是SR图像与GT一起输入Discriminator来得到loss,再将不同的loss进行加权组合。并且Generator和Discriminator的参数都需要更新。

\begin{minted}[xleftmargin=20pt,linenos,bgcolor=bg]{python}
nondist_validation(self, dataloader, current_iter, tb_logger, save_img)
\end{minted}

训练中进行验证。会停止训练,调用test(),进行验证并记录指标、保存结果图,然后继续训练。

其他model的思想与SR Model大同小异,都是相互继承公用的函数,单独改写有特别需求的函数,完成一个模型包括获得数据、更新参数、验证等步骤的训练过程。


\subsection{自定义模型,以SRGAN Model为例}

当需要自定义某个模型的时候,首先在/basicsr/models下新建XXX\_model.py文件,根据需要继承Base Model或其他model,然后根据自己的需求改写主要函数即可。

比如SRGAN\_Model继承了绝大多数SR\_Model的函数,但是改写了optimize\_parameters()。而RealSRGAN\_Model又继承了SRGAN\_Model,改写了feed\_data()以在线生成带有不同degradation的训练数据。

\begin{note} % ---------------- Note block ---------------- %
	上述被改写函数的介绍,参见章节\ref{Model:SR Model}。
\end{note}



\section{损失函数(Loss)}

\section{训练(优化器与学习率调度器)}

\section{算子}

\section{日志系统 Logger}

- 大概讲解

- 日志的各项什么含义

\end{document}