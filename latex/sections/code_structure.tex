\documentclass[../main.tex]{subfiles}

\begin{document}

\chapter{代码主体结构}
\vspace{-2cm}

简要总结

\section{整体框架}

- 参考 \url{https://github.com/XPixelGroup/BasicSR/blob/master/docs/DesignConvention_CN.md}

- 代码接口在 http://basicsr.readthedocs.io/,后续看看能否作为附录存在

%##################################################################################################
\begin{figure}[t]
    %\vspace{-0.5cm}
    \begin{center}
        %\fbox{\rule{0pt}{2.5in} \rule{0.9\linewidth}{0pt}}
        \includegraphics[width=\linewidth]{figures/basicsr_logo.png}
        \vspace{-1cm}
        \caption{图标题。BasicSR Logo。}
        \label{fig:logo}
    \end{center}
    %\vspace{-0.7cm}
\end{figure}
%##################################################################################################

\section{配置(Options)与注册器(Register)}

- 态实例化与REGISTER注册机制

- 约定

- 如何避免重复的类名和函数名

\begin{minted}[xleftmargin=20pt,linenos,bgcolor=bg]{python}
class RepConv(nn.Module):
    """Re-parameterizable block for RepSR."""

    def __init__(self,
                 in_channels,
                 out_channels,
                 kernel_size,
                 stride=1,
                 padding=0,
                 dilation=1,
                 groups=1,
                 padding_mode='zeros',
                 deploy=False,
                 width_multiplier=2,
                 with_bn=True,
                 frozen_bn=False):
        super(RepConv, self).__init__()
        self.deploy = deploy
        self.in_channels = in_channels
        self.out_channels = out_channels
        self.kernel_size = kernel_size
        self.stride = stride
        self.padding = padding
        self.dilation = dilation
        self.groups = groups
        self.with_bn = with_bn
        self.frozen_bn = frozen_bn

        self.mid_channels = out_channels * width_multiplier
        self.rep_c1_2 = nn.Conv2d(
            in_channels=self.mid_channels,
            out_channels=out_channels,
            kernel_size=1,
            stride=stride,
            padding=0,
            groups=groups,
            bias=True)

        # initialization
        init_list = [
            self.rep_identity, self.rep_c3_1, self.rep_c1_1, self.rep_c3_2,
            self.rep_c1_2
        ]
        if with_bn:
            init_list.extend([self.rep_bn_1, self.rep_bn_2])
        default_init_weights(init_list, scale=0.1)

    def forward(self, inputs, frozen_bn=None):
        if frozen_bn is None:
            frozen_bn = self.frozen_bn

        if hasattr(self, 'rep_merge'):
            return self.rep_merge(inputs)
        if self.with_bn:
            id_out = self.rep_identity(inputs)
            out_1 = self.rep_c1_1(
                self.rep_bn_1(self.rep_c3_1(inputs), frozen_bn))
            out_2 = self.rep_c1_2(
                self.rep_bn_2(self.rep_c3_2(inputs), frozen_bn))
        else:
            id_out = self.rep_identity(inputs)
            out_1 = self.rep_c1_1(self.rep_c3_1(inputs))
            out_2 = self.rep_c1_2(self.rep_c3_2(inputs))

        return id_out + out_1 + out_2
\end{minted}


\section{数据(Data Loader)}

\section{网络结构(Architecture)}

\section{模型(Model)}

- Base模型

- 其他模型

- 继承关系

\section{损失函数(Loss)}

\section{训练(优化器与学习率调度器)}

\section{算子}

\section{日志系统 Logger}

- 大概讲解

- 日志的各项什么含义

\end{document}