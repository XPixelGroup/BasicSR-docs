\documentclass[../main.tex]{subfiles}

\begin{document}

\chapter{数据准备}
\vspace{-2cm}
这部分主要讲述数据存储形式,file client类的基本构成,以及各种数据集的获取和描述。


% ------------------------------------------------------------------------------
\section{常见用法}\label{data_preparation:common_use}

目前, 我们可以通过configuration.yaml文件方便的修改. 以支持DIV2K的 \href{https://github.com/XPixelGroup/BasicSR/blob/master/basicsr/data/paired_image_dataset.py}{PairedImageDataset} 为例, 根据不同的要求修改yaml文件.

\begin{enumerate}
\item 直接读取硬盘数据
\begin{minted}[xleftmargin=20pt,linenos,bgcolor=bg]{yaml}
type: PairedImageDataset
dataroot_gt: datasets/DIV2K/DIV2K_train_HR_sub
dataroot_lq: datasets/DIV2K/DIV2K_train_LR_bicubic/X4_sub
io_backend:
  type: disk
\end{minted}

\item 使用LMDB. 在使用前需要先制作LMDB, 参见 \href{https://github.com/XPixelGroup/BasicSR/blob/master/docs/DatasetPreparation_CN.md#LMDB%E5%85%B7%E4%BD%93%E8%AF%B4%E6%98%8E}{LMDB具体说明}, 注意我们在原有的 LMDB 上, 新增加了 meta 信息, 而且具体保存二进制内容也不同, 因此其他来源的LMDB并不能直接拿过来使用
\begin{minted}[xleftmargin=20pt,linenos,bgcolor=bg]{yaml}
type: PairedImageDataset
dataroot_gt: datasets/DIV2K/DIV2K_train_HR_sub.lmdb
dataroot_lq: datasets/DIV2K/DIV2K_train_LR_bicubic_X4_sub.lmdb
io_backend:
  type: lmdb
\end{minted}

\item 使用Memcached 机器/集群需要支持 Memcached. 具体的配置文件根据实际的 Memcached 需要进行修改:
能直接拿过来使用
\begin{minted}[xleftmargin=20pt,linenos,bgcolor=bg]{yaml}
type: PairedImageDataset
dataroot_gt: datasets/DIV2K_train_HR_sub
dataroot_lq: datasets/DIV2K_train_LR_bicubicX4_sub
io_backend:
  type: memcached
  server_list_cfg: /mnt/lustre/share/memcached_client/server_list.conf
  client_cfg: /mnt/lustre/share/memcached_client/client.conf
  sys_path: /mnt/lustre/share/pymc/py3
\end{minted}
\end{enumerate}


% ------------------------------------------------------------------------------
\section{数据存储格式}\label{data_preparation:data_format}

% - 参考: \url{https://github.com/XPixelGroup/BasicSR/blob/master/docs/DatasetPreparation_CN.md}

目前支持的数据存储形式有以下三种:
\begin{enumerate}
\item直接以图像/视频帧的格式存放在硬盘
\item制作成 LMDB. 训练数据使用这种形式, 一般会加快读取速度
\item若是支持 Memcached, 则可以使用. 它们一般应用在集群上
\end{enumerate}

% ------------------------------------------------------------------------------
\subsection{LMDB具体说明}

我们在训练的时候使用 LMDB 存储形式可以加快IO和CPU解压缩的速度 (测试的时候数据较少, 一般就没有太必要使用 LMDB). 其具体的加速要根据机器的配置来, 以下几个因素会影响:
\begin{enumerate}
\item有的机器设置了定时清理缓存, 而 LMDB 依赖于缓存. 因此若一直缓存不进去, 则需要检查一下. 一般 free -h 命令下, LMDB 占用的缓存会记录在 buff/cache 条目下面
\item机器的内存是否足够大, 能够把整个 LMDB 数据都放进去. 如果不是, 则它由于需要不断更换缓存, 会影响速度
\item若是第一次缓存 LMDB 数据集, 可能会影响训练速度. 可以在训练前, 进入 LMDB 数据集目录, 把数据先缓存进去: cat data.mdb > /dev/nul
\end{enumerate}

% ------------------------------------------------------------------------------
\subsection{文件结构}

除了标准的 LMDB 文件 (data.mdb 和 lock.mdb) 外, 我们还增加了
 meta\_info.txt 来记录额外的信息. 下面用一个例子来说明.

\begin{minted}[xleftmargin=20pt,linenos,bgcolor=bg]{yaml}
DIV2K_train_HR_sub.lmdb
├── data.mdb
├── lock.mdb
├── meta_info.txt
\end{minted}

% ----------------------------------
\subsection{meta信息}

meta\_info.txt, 我们采用txt来记录, 是为了可读性. 其里面的内容为:
\begin{minted}[xleftmargin=20pt,linenos,bgcolor=bg]{yaml}
0001_s001.png (480,480,3) 1
0001_s002.png (480,480,3) 1
0001_s003.png (480,480,3) 1
0001_s004.png (480,480,3) 1
...
\end{minted}

每一行记录了一张图片, 有三个字段, 分别表示:
\begin{enumerate}
\item图像名称 (带后缀): 0001\_s001.png
\item图像大小: (480,480,3) 表示是480x480x3的图像
\item其他参数 (BasicSR里面使用了 cv2 压缩 png 程度): 因为在复原任务中, 我们通常使用 png 来存储, 所以这个 1 表示 png 的压缩程度
CV\_IMWRITE\_PNG\_COMPRESSION 是 1. 它可以取值为 [0, 9] 的整数, 更大的值表示更强的压缩, 即更小的储存空间和更长的压缩时间
\end{enumerate}

% ----------------------------------
\subsection{二进制内容}

为了方便, 我们在 LMDB 数据集中存储的二进制内容是 cv2 encode 过的 image: cv2.imencode(`.png`, img, [cv2.IMWRITE\_PNG\_COMPRESSION, compress\_level]. 可以通过 compress\_level 控制压缩程度, 平衡存储空间和读取(包括解压缩)的速度.

% ----------------------------------
\subsection{如何制作}

我们提供了脚本来制作. 在运行脚本前, 需要根据需求修改相应的参数. 目前支持 DIV2K, REDS 和 Vimeo90K 数据集; 其他数据集可仿照进行制作.
python scripts/data\_preparation/create\_lmdb.py

% ----------------------------------
\subsection{预读取数据}

除了使用LMDB来加速外, 还可以采用预读取数据来加速, 实现参见\href{https://github.com/XPixelGroup/BasicSR/blob/master/basicsr/data/prefetch_dataloader.py}{prefetch\_dataloader}
这个可以通过配置文件中的 prefetch\_mode 来指定. 目前提供了三种模式:

\begin{enumerate}
\item None. 默认不使用. 如果使用了 LMDB 或者 IO 不成问题, 则可不使用
\begin{minted}[xleftmargin=20pt,linenos,bgcolor=bg]{yaml}
prefetch_mode: ~
\end{minted}

\item prefetch\_mode: cuda. 使用 CUDA prefetcher, 具体介绍参见 NVIDIA/apex. 它会多占用一些GPU显存. 注意: 这个模式下, 一定要设置 pin\_memory=True
\begin{minted}[xleftmargin=20pt,linenos,bgcolor=bg]{yaml}
prefetch_mode: cuda
pin_memory: true
\end{minted}

\item prefetch\_mode: cpu. 使用 CPU prefetcher, 具体介绍参见 IgorSusmelj/pytorch-styleguide. (目前测试,这个加速不明显)
\begin{minted}[xleftmargin=20pt,linenos,bgcolor=bg]{yaml}
prefetch_mode: cpu
num_prefetch_queue: 1  # 1 by default
\end{minted}

\end{enumerate}



% ----------------------------------
\section{File Client 介绍}\label{data_preparation:file_client}

我们参考了 MMCV 优雅的 FileClient 设计. 为了使其兼容 BasicSR, 我们对接口做了一些改动 (主要是为了适应LMDB), 参见 \href{https://github.com/XPixelGroup/BasicSR/blob/master/basicsr/utils/file_client.py}{file\_client.py}.

\begin{minted}[xleftmargin=20pt,linenos,bgcolor=bg]{python}
class FileClient(object):
    """A general file client to access files in different backend.
    The client loads a file or text in a specified backend from its path
    and return it as a binary file. it can also register other backend
    accessor with a given name and backend class.
    Attributes:
        backend (str): The storage backend type. Options are "disk",
            "memcached" and "lmdb".
        client (:obj:`BaseStorageBackend`): The backend object.
    """

    _backends = {
        'disk': HardDiskBackend,
        'memcached': MemcachedBackend,
        'lmdb': LmdbBackend,
    }

    def __init__(self, backend='disk', **kwargs):
        if backend not in self._backends:
            raise ValueError(f'Backend {backend} is not supported.'
                             f' Currently supported ones are'
                             f'{list(self._backends.keys())}')
        self.backend = backend
        self.client = self._backends[backend](**kwargs)

    def get(self, filepath, client_key='default'):
        # client_key is used only for lmdb, where different fileclients have
        # different lmdb environments.
        if self.backend == 'lmdb':
            return self.client.get(filepath, client_key)
        else:
            return self.client.get(filepath)

    def get_text(self, filepath):
        return self.client.get_text(filepath)
\end{minted}


% ----------------------------------
\section{常见数据集介绍与准备}\label{data_preparation:dataset}

推荐把数据通过 ln -s xxx yyy 软链到 BasicSR/datasets 下. 如果你的文件结构不同, 需要相应地修改 configuration yaml 文件的路径.

% ----------------------------------
\subsection{DIV2K 与 DF2K}

DIV2K与DF2K数据集被广泛使用在图像复原的任务中.

\noindent\textbf{数据准备步骤}
\begin{enumerate}
\item 从\href{https://data.vision.ee.ethz.ch/cvl/DIV2K}{官网}下载数据
\item Crop to sub-images: 因为 DIV2K 数据集是 2K 分辨率的 (比如: 2048x1080), 而我们在训练的时候往往并不要那么大 (常见的是 128x128 或者 192x192 的训练patch). 因此我们可以先把 2K 的图片裁剪成有 overlap 的 480x480 的子图像块. 然后再由 dataloader 从这个 480x480 的子图像块中随机 crop 出 128x128 或者 192x192 的训练 patch.
运行脚本 \href{https://github.com/XPixelGroup/BasicSR/blob/master/scripts/data_preparation/extract_subimages.py}{extract\_subimages.py}:
\begin{minted}[xleftmargin=20pt,linenos,bgcolor=bg]{yaml}
python scripts/data_preparation/extract_subimages.py
\end{minted}
使用之前可能需要修改文件里面的路径和配置参数. 注意: sub-image 的尺寸和训练 patch 的尺寸 (gt\_size) 是不同的. 我们先把2K分辨率的图像 crop 成 sub-images (往往是 480x480), 然后存储起来. 在训练的时候, dataloader 会读取这些 sub-images, 然后进一步随机裁剪成 gt\_size x gt\_size 的大小.
\item\,[可选]若需要使用 LMDB, 则需要制作 LMDB, 参考 LMDB 具体说明. python scripts/data\_preparation/create\_lmdb.py, 注意选择 create\_lmdb\_for\_div2k 函数, 并需要修改函数相应的配置和路径.
\item 测试: tests/test\_paired\_image\_dataset.py, 注意修改函数相应的配置和路径.
\item\,[可选] 若需要使用 meta\_info\_file,
运行 python scripts/data\_preparation/\\generate\_meta\_info.py 来生成 meta\_info\_file.
\end{enumerate}

% ----------------------------------
\subsection{REDS}

REDS 数据集官方网站:\href{https://seungjunnah.github.io/Datasets/reds.html}{REDS}.
我们重新整合了 training 和 validation 数据到一个文件夹中: 训练集合原来有240个 clip (序号从000到239), 我们把 validation clips 重命名, 从240到269.\\

\noindent\textbf{Validation 的划分}

官方的 validation 划分和 EDVR 的划分不同 (当时为了比赛的设置):

\begin{table}[htbp]
    \centering
        \begin{tabular}{|c|c|c|}
            \hline
            \textbf{name} & \textbf{clips} & \textbf{total number}                               \\ \hline
            REDSOfficial              & [240, 269]      & 30 clips                               \\ \hline
            REDS4          & 000, 011, 015, 020 clips from the original training set      & 4 clips       \\ \hline
        \end{tabular}
    \caption{Validation 的划分}
\end{table}
余下的 clips 拿来做训练集合. 注意: 我们不需要显式地分开训练和验证集合, dataloader 会做这件事.\\

\noindent\textbf{数据准备步骤}

\begin{enumerate}
\item 从\href{https://seungjunnah.github.io/Datasets/reds.html}{官网}下载数据
\item 整合 training 和 validation 数据: python scripts/data\_preparation/regroup\_reds\_dataset.py
\item\,[可选] 若需要使用 LMDB, 则需要制作 LMDB, 参考 LMDB 具体说明. python scripts/data\_preparation/create\_lmdb.py, 注意选择 create\_lmdb\_for\_reds 函数, 并需要修改函数相应的配置和路径.
\item 测试: python tests/test\_reds\_dataset.py, 注意修改函数相应的配置和路径.
\end{enumerate}

% ----------------------------------
\subsection{Vimeo90K}

官网地址:\href{http://toflow.csail.mit.edu/}{Vimeo90K}\\

\noindent\textbf{数据准备步骤}
\begin{enumerate}
\item 下载数据: \href{http://data.csail.mit.edu/tofu/dataset/vimeo_septuplet.zip}{Septuplets dataset --> The original training + test set (82GB)}. 这些是 Ground-Truth. 里面有 sep\_trainlist.txt 文件来区分训练数据.
\item 生成低分辨率图片. The low-resolution images in the Vimeo90K test dataset are generated with the MATLAB bicubic downsampling kernel. Use the script \textbf{data\_scripts/generate\_LR\_Vimeo90K.m} (run in MATLAB) to generate the low-resolution images.
\item\,[可选] 若需要使用 LMDB, 则需要制作 LMDB, 参考 LMDB 具体说明. 运行脚本: python scripts/data\_preparation/create\_lmdb.py, 注意选择 create\_lmdb\_for\_vimeo90k 函数, 并需要修改函数相应的配置和路径.
\item 测试: python tests/test\_vimeo90k\_dataset.py, 注意修改函数相应的配置和路径.
\end{enumerate}

\end{document}