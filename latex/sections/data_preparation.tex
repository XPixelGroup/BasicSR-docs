\documentclass[../main.tex]{subfiles}

\begin{document}

\chapter{数据准备}\label{chapter:data_preparation}
\vspace{-2cm}
这部分主要讲述数据存储形式,FileClient 类,以及一些常见数据集的获取和描述。

% ------------------------------------------------------------------------------
\section{常见用法}\label{data_preparation:common_use}

目前支持的数据存储形式有以下三种:

\begin{enumerate}
    \item 直接以图像/视频帧的格式存放在硬盘
    \item 制作成 LMDB. 训练数据使用这种形式, 一般会加快读取速度
    \item 若是支持 Memcached, 则可以使用. 它们一般应用在集群上
\end{enumerate}

目前,我们可以通过 yaml 配置文件方便地修改。以支持 DIV2K 的 \href{https://github.com/XPixelGroup/BasicSR/blob/master/basicsr/data/paired_image_dataset.py}{PairedImageDataset} 为例,根据不同的要求修改 yaml 文件。

\begin{enumerate}
    \item 直接读取硬盘数据
          \begin{minted}[xleftmargin=20pt,linenos,bgcolor=bg]{yaml}
type: PairedImageDataset
dataroot_gt: datasets/DIV2K/DIV2K_train_HR_sub
dataroot_lq: datasets/DIV2K/DIV2K_train_LR_bicubic/X4_sub
io_backend:
  type: disk
\end{minted}

    \item 使用 LMDB。在使用前需要先制作 LMDB, 参见章节\ref{data_preparation:lmdb}:\nameref{data_preparation:lmdb}, 注意我们在原有的 LMDB 上, 新增加了 meta 信息, 而且具体保存二进制内容也不同, 因此其他来源的 LMDB 并不能直接拿过来使用
          \begin{minted}[xleftmargin=20pt,linenos,bgcolor=bg]{yaml}
type: PairedImageDataset
dataroot_gt: datasets/DIV2K/DIV2K_train_HR_sub.lmdb
dataroot_lq: datasets/DIV2K/DIV2K_train_LR_bicubic_X4_sub.lmdb
io_backend:
  type: lmdb
\end{minted}

    \item 使用 Memcached。 机器/集群需要支持 Memcached。具体的配置文件根据实际的 Memcached 需要进行修改:
          能直接拿过来使用
          \begin{minted}[xleftmargin=20pt,linenos,bgcolor=bg]{yaml}
type: PairedImageDataset
dataroot_gt: datasets/DIV2K_train_HR_sub
dataroot_lq: datasets/DIV2K_train_LR_bicubicX4_sub
io_backend:
  type: memcached
  server_list_cfg: /mnt/lustre/share/memcached_client/server_list.conf
  client_cfg: /mnt/lustre/share/memcached_client/client.conf
  sys_path: /mnt/lustre/share/pymc/py3
\end{minted}
\end{enumerate}

% ------------------------------------------------------------------------------
\section{数据存储格式}\label{data_preparation:data_format}

% ----------------------------------
\subsection{LMDB 具体说明}\label{data_preparation:lmdb}

我们在训练的时候使用 LMDB 存储形式可以加快 IO 和 CPU 解压缩的速度 (测试的时候数据较少, 一般就没有太必要使用 LMDB)。其具体的加速要根据机器的配置来,以下几个因素会影响:
\begin{enumerate}
    \item 有的机器设置了定时清理缓存, 而 LMDB 依赖于缓存. 因此若一直缓存不进去, 则需要检查一下. 一般 \texttt{free -h} 命令下, LMDB 占用的缓存会记录在 \texttt{buff/cache} 条目下面
    \item 机器的内存是否足够大, 能够把整个 LMDB 数据都放进去. 如果不是, 则它由于需要不断更换缓存, 会影响速度
    \item 若是第一次缓存 LMDB 数据集, 可能会影响训练速度. 可以在训练前, 进入 LMDB 数据集目录, 把数据先缓存进去: \texttt{cat data.mdb > /dev/null}
\end{enumerate}

\subsubsection{文件结构}

除了标准的 LMDB 文件 (data.mdb 和 lock.mdb) 外,我们还增加了 meta\_info.txt 来记录额外的信息。下面用一个例子来说明:

\begin{minted}[xleftmargin=20pt,linenos,bgcolor=bg]{yaml}
DIV2K_train_HR_sub.lmdb
├── data.mdb
├── lock.mdb
├── meta_info.txt
\end{minted}

\subsubsection{meta信息}

meta\_info.txt, 我们采用 txt 来记录, 是为了可读性。其里面的内容为:

\begin{minted}[xleftmargin=20pt,linenos,bgcolor=bg]{yaml}
0001_s001.png (480,480,3) 1
0001_s002.png (480,480,3) 1
0001_s003.png (480,480,3) 1
0001_s004.png (480,480,3) 1
...
\end{minted}

每一行记录了一张图片, 有三个字段, 分别表示:
\begin{enumerate}
    \item 图像名称 (带后缀): 0001\_s001.png
    \item 图像大小: (480,480,3) 表示是$480\times480\times3$的图像
    \item 其他参数 (BasicSR里面使用了 cv2 压缩 png 程度): 因为在复原任务中, 我们通常使用 png 来存储, 所以这个 1 表示 png 的压缩程度
          CV\_IMWRITE\_PNG\_COMPRESSION 是 1. 它可以取值为 [0, 9] 的整数, 更大的值表示更强的压缩, 即更小的储存空间和更长的压缩时间
\end{enumerate}

\subsubsection{二进制内容}

为了方便, 我们在 LMDB 数据集中存储的二进制内容是 cv2 encode 过的 image: cv2.imencode(`.png`, img, [cv2.IMWRITE\_PNG\_COMPRESSION, compress\_level]. 可以通过 compress\_level 控制压缩程度, 平衡存储空间和读取(包括解压缩)的速度.

\subsubsection{如何制作}

我们提供了脚本 \href{https://github.com/XPixelGroup/BasicSR/blob/master/scripts/data_preparation/create_lmdb.py}{scripts/data\_preparation/create\_lmdb.py} 来制作. 在运行脚本前, 需要根据需求修改相应的参数. 目前支持 DIV2K, REDS 和 Vimeo90K 数据集; 其他数据集可仿照进行制作。

\begin{minted}[xleftmargin=20pt,bgcolor=bg]{bash}
python scripts/data_preparation/create_lmdb.py  --dataset div2k
python scripts/data_preparation/create_lmdb.py  --dataset reds
python scripts/data_preparation/create_lmdb.py  --dataset vimeo90k
\end{minted}

\begin{note} % ---------------- Note block ---------------- %
    \textbf{加速 IO 方法}

    除了使用 LMDB 加速 IO 外,还可以使用 prefetch 方式,具体参见章节\ref{code_structure:dataset_prefecth}:\nameref{code_structure:dataset_prefecth}。
\end{note}

% ----------------------------------
\section{File Client 介绍}\label{data_preparation:file_client}

我们参考了 MMCV 的 FileClient 设计。为了使其兼容 BasicSR,我们对接口做了一些改动 (主要是为了适应 LMDB)。具体可以参见代码 \href{https://github.com/XPixelGroup/BasicSR/blob/master/basicsr/utils/file_client.py}{file\_client.py}。

\todo{后续添加详细说明}

% ----------------------------------
\section{常见数据集介绍与准备}\label{data_preparation:dataset}

推荐把数据通过 \texttt{ln -s src dst} 软链到 datasets 目录下。

% ----------------------------------
\subsection{图像数据集 DIV2K 与 DF2K}

DIV2K 与 DF2K 数据集被广泛使用在图像复原的任务中。
其中 DF2K 是 DIV2K 和 Flickr2K 的融合。

\noindent\textbf{数据准备步骤}
\begin{enumerate}
    \item 从 \href{https://data.vision.ee.ethz.ch/cvl/DIV2K}{DIV2K 官网}下载数据。 Flickr 2K 可从 \url{https://cv.snu.ac.kr/research/EDSR/Flickr2K.tar} 下载
    \item Crop to sub-images: 因为 DIV2K 数据集是 2K 分辨率的 (比如: 2048x1080), 而我们在训练的时候往往并不要那么大 (常见的是 128$\times$128 或者 192$\times$192 的训练 patch). 因此我们可以先把 2K 的图片裁剪成有 overlap 的 480$\times$480 的子图像块. 然后再由 dataloader 从这个 480$\times$480 的子图像块中随机 crop 出 128$\times$128 或者 192$\times$192 的训练 patch。
          运行脚本 \href{https://github.com/XPixelGroup/BasicSR/blob/master/scripts/data_preparation/extract_subimages.py}{extract\_subimages.py}:
          \begin{minted}[xleftmargin=20pt,bgcolor=bg]{bash}
python scripts/data_preparation/extract_subimages.py
\end{minted}
          使用之前可能需要修改文件里面的路径和配置参数. 注意: sub-image 的尺寸和训练 patch 的尺寸 (gt\_size) 是不同的. 我们先把2K分辨率的图像 crop 成 sub-images (往往是 480$\times$480), 然后存储起来. 在训练的时候, dataloader 会读取这些 sub-images, 然后进一步随机裁剪成 gt\_size $\times$ gt\_size 的大小
    \item\,[可选] 若需要使用 LMDB, 则需要制作 LMDB, 参考章节\ref{data_preparation:lmdb}:\nameref{data_preparation:lmdb}
          运行脚本:
          \begin{minted}[xleftmargin=20pt,bgcolor=bg]{bash}
python scripts/data_preparation/create_lmdb.py --dataset div2k
\end{minted}
          注意选择 create\_lmdb\_for\_div2k 函数, 并需要修改函数相应的配置和路径
    \item 单元测试: 我们可以单独测试 dataset 是否 正常。test\_scripts/test\_paired\_image\_dataset.py, 注意修改函数相应的配置和路径.
    \item\,[可选] 若需要生成 meta\_info\_file 文件,
          运行
          \begin{minted}[xleftmargin=20pt,bgcolor=bg]{bash}
python scripts/data_preparation/generate_meta_info.py
\end{minted}
\end{enumerate}

% ----------------------------------
\subsection{视频帧数据集 REDS}

REDS 是常用的视频帧数据集。数据集官方网站:\url{https://seungjunnah.github.io/Datasets/reds.html}。
我们重新整合了 training 和 validation 数据到一个文件夹中: 训练集合原来有240个 clip (序号从000到239), 我们把 validation clips 重命名, 从240到269。

\noindent\textbf{Validation 的划分}

官方的 validation 划分和 EDVR、BasicVSR 论文中的划分不同 (当时为了比赛的设置):

\begin{table}[h]
    \centering
    \begin{tabular}{|c|c|c|}
        \hline
        \textbf{name} & \textbf{clips}                                          & \textbf{total number} \\ \hline
        REDSOfficial  & [240, 269]                                              & 30 clips              \\ \hline
        REDS4         & 000, 011, 015, 020 clips from the original training set & 4 clips               \\ \hline
    \end{tabular}
    \caption{REDS 数据集中 Validation 的划分}
\end{table}
余下的 clips 拿来做训练集合。 注意: 我们不需要显式地分开训练和验证集合, dataloader 会做这件事.

\noindent\textbf{数据准备步骤}

\begin{enumerate}
    \item 从\href{https://seungjunnah.github.io/Datasets/reds.html}{官网}下载数据
    \item 整合 training 和 validation 数据。运行
    \begin{minted}[xleftmargin=20pt,bgcolor=bg]{bash}
python scripts/data_preparation/regroup_reds_dataset.py
\end{minted}
    \item\,[可选] 若需要使用 LMDB, 则需要制作 LMDB, 参考章节\ref{data_preparation:lmdb}:\nameref{data_preparation:lmdb}。运行
    \begin{minted}[xleftmargin=20pt,bgcolor=bg]{bash}
 python scripts/data_preparation/create_lmdb.py --dataset reds
\end{minted}
    注意选择 create\_lmdb\_for\_reds 函数, 并需要修改函数相应的配置和路径
    \item 单元测试: 我们可以单独测试 dataset 是否 正常。test\_scripts/test\_reds\_dataset.py, 注意修改函数相应的配置和路径
\end{enumerate}

% ----------------------------------
\subsection{视频帧数据集 Vimeo90K}

Vimeo90K 是常用的视频帧数据集。官网地址:\url{http://toflow.csail.mit.edu}

\noindent\textbf{数据准备步骤}
\begin{enumerate}
    \item 下载数据: \href{http://data.csail.mit.edu/tofu/dataset/vimeo_septuplet.zip}{Septuplets dataset --> The original training + test set (82GB)}. 这些是 Ground-Truth. 里面有 sep\_trainlist.txt 文件来区分训练数据
    \item 生成低分辨率图片。Vimeo90K测试集中的低分辨率图片是由 MATLAB bicubic 降采样函数而来。运行脚本 \texttt{data\_scripts/generate\_LR\_Vimeo90K.m} (run in MATLAB) 来生成低清图片
    \item [可选] 若需要使用 LMDB, 则需要制作 LMDB, 参考章节\ref{data_preparation:lmdb}:\nameref{data_preparation:lmdb}。运行
    \begin{minted}[xleftmargin=20pt,bgcolor=bg]{bash}
 python scripts/data_preparation/create_lmdb.py --dataset vimeo90k
\end{minted}
    注意选择 create\_lmdb\_for\_vimeo90k 函数, 并需要修改函数相应的配置和路径
    \item 单元测试: 我们可以单独测试 dataset 是否 正常。test\_scripts/test\_vimeo90k\_dataset.py, 注意修改函数相应的配置和路径
\end{enumerate}

\end{document}