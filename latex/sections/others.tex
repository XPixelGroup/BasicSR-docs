\documentclass[../main.tex]{subfiles}

\begin{document}

\chapter{其他}
\vspace{-2cm}

在这个章节中,我们将对 BasicSR 其他的一些常用的设计进行说明,比如 Debug 模式, 自动 Resume, 命令行修改参数等等。

\section{Debug 模式}\label{others:debug_mode}

正式训练之前,你可以用 debug 模式检查是否正常运行。在 debug 模式下:
\begin{itemize}
    \item 程序会在每次 iteration 下都打印日志,并且经过8次 iterations 后,便会进入 validation 阶段。这样可以快速方便地查看代码是否可以正常运行,而不用实际训练。毕竟实际训练可能很慢,等半天后,发现程序崩溃了,而原因是 validation 中有 bug。
    \item 在 debug 模式中,并不会使用 tensorboard logger 和 wandb logger,以保证日志文件的简洁性。
\end{itemize}

\textbf{如何进入 debug 模式?}

\begin{enumerate}[方式 1.]
    \item 在命令行最后加入 ‘\textbf{-{}-debug}’。比如:
          \begin{minted}[xleftmargin=20pt,linenos,breaklines,bgcolor=bg]{python}
 python basicsr/train.py -opt options/train/SRResNet_SRGAN/train_MSRResNet_x4.yml --debug
    \end{minted}
    \item 在配置文件 yml 文件的 name 中添加 ‘debug’ 字符。只要在实验名字中有 ‘\uline{debug}’ 字样, 则会进入 debug 模式。
\end{enumerate}

\section{自动 resume}\label{others:auto_resume}

\todo{Resume 加入第四章节的训练部分}
当遇到程序中断的时候,我们希望程序能够自动 resume。BasicSR 当然提供了这样的功能啦。

只要在命令行中加入 ‘\textbf{-{}-auto\_resume}’,程序就会找到保存的最近的模型参数和状态,并加载进来,接着训练啦。

\section{命令行修改配置}\label{others:yml_modification_with_commands}

BasicSR 使用 yml 文件进行配置。我们也推荐这样的方式,因为这样可以记录并跟踪每一个实验的配置。
但我们也希望在仅仅修改了一个小配置的情况下 (比如修改 random seed),不需要麻烦地新建并修改 yml 配置文件。它可以使用原先的 yml 配置文件,而在命令行中对配置做出修改。

BasicSR 提供了一个简便的命令行参数 ‘\textbf{-{}-force\_yml}’,在命令行中它的用法如下:

\begin{enumerate}
    \item ‘-{}-force\_yml’ 后面接如下的字符串,每一个字符串修改一个配置,如果有多个配置需要修改,以空格隔开多个字符串。
    \item 字符串采用格式:‘train:ema\_decay=0.999’。等号 (=) 前后分别表示 key 和 value。如果有层级结构,使用冒号 (:) 来区分。
    \item 修改之后记得检查打印的日志是否符合预期哈!
\end{enumerate}

\begin{exampleBox}[]{示例}

    有如下的配置 yml 文件,我们希望修改:1) random seed 为 1; 2) ema\_decay 为0.5; 3) 名字中体现配置。

    \begin{minted}[xleftmargin=20pt,linenos,breaklines,bgcolor=bg]{yaml}
# general settings
name: 001_MSRResNet_x4_f64b16_DIV2K_1000k_B16G1_wandb
model_type: SRModel
scale: 4
manual_seed: 0

...

train:
  ema_decay: 0.999
    \end{minted}

    那么,我们的命令行就变成了:
    \begin{minted}[xleftmargin=20pt,linenos,breaklines,bgcolor=bg]{bash}
python basicsr/train.py -opt options/train/SRResNet_SRGAN/train_MSRResNet_x4.yml --force_yml manual_seed=1 train:ema_decay=0.5 name=001_MSRResNet_x4_f64b16_DIV2K_1000k_B16G1_rand1_ema_decay0.5
    \end{minted}
\end{exampleBox}


\section{EMA 介绍}\label{others:ema}

EMA (Exponential Moving Average),指数移动平均。
它是用来“平均”一个变量在历史上的值。使用怎样的权重平均呢?如名字所示,随着时间,越是过往的时间,以一个指数衰减的权重来平均。

在 BasicSR 里面,EMA 一般作用在模型的参数上。它的效果一般是:

\begin{itemize}
    \item 稳定训练效果。GAN 训练的结果一般瑕疵更少,视觉效果更好
    \item 对于以 PSNR 为目的的模型,其 PSNR 一般会更高一些。
\end{itemize}

由于开启 EMA的代价几乎可以不计,所以我们推荐开启 EMA。

\begin{hl} % ---------------- Highlight block ---------------- %
    \textbf{如何开启 EMA?}

    在 yml 的配置文件中,我们只要指定 ema\_decay 大于 0,就会开始 EMA。


    \begin{minted}[xleftmargin=20pt,linenos,breaklines,bgcolor=bg]{yaml}
# training settings
train:
  ema_decay: 0.9      # 开启 EMA, 滑动系数为 0.9
\end{minted}
\end{hl}

开启了 EMA 后,保存的模型会有两个字段: \texttt{params} 和 \texttt{params\_ema},其中  \texttt{params\_ema} 就是 EMA 保存的模型。
在测试或者推理的时候,我们要留意加载的到底是 \texttt{params} 还是 \texttt{params\_ema},这个在 yml 文件中,一般通过  \texttt{param\_key\_g: params\_ema} 来指定。

% \section{多卡(多机多卡)训练}
% 放到 3.2 中

\section{单元测试}

我们使用单元测试主要保证输入输出 shape 的正确性,以及一些流程的正确定。
一般来说,我们较少使用。BasicSR为了完备,将单元测试也加入进来。

在 \texttt{tests} 目录的函数会被单元测试执行。
单元测试需要 GPU CUDA 环境。


\begin{hl} % ---------------- Highlight block ---------------- %
    \textbf{单元测试命令}


    \begin{minted}[xleftmargin=20pt,linenos,breaklines,bgcolor=bg]{bash}
python -m pytest tests/
    \end{minted}
\end{hl}



\end{document}