\documentclass[../main.tex]{subfiles}

\begin{document}


\chapter{安装}
\vspace{-2cm}

本章节介绍了两种不同的安装 BasicSR 方式:本地 clone 和通过 pip 安装;除此之外,我们将安装过程中有可能会遇到的一些问题进行了汇总。

\section{环境要求}

\begin{enumerate}
    \item Python 版本 >= 3.7 (推荐使用\href{https://www.anaconda.com/products/distribution#linux}{Anaconda}或者\href{https://docs.conda.io/en/latest/miniconda.html}{Miniconda})
    \item \href{https://pytorch.org/}{PyTorch 版本 >= 1.7}
    \item NVIDIA GPU + \href{https://developer.nvidia.com/cuda-downloads}{CUDA}
    \item \href{https://pytorch.org/}{Torchvision 版本 >= 0.9.0} (推荐)
\end{enumerate}

\section{BASICSR\_EXT 和 BASICSR\_JIT 环境变量介绍}
\label{section:env}

如果项目中不需要使用以下的 PyTorch C++ 扩展,则此小节可以跳过:

\begin{itemize}
    \item 可变性卷积 (如果安装的 Torchvision 版本 >= 0.9.0,可跳过):\href{https://github.com/XPixelGroup/BasicSR/tree/master/basicsr/ops}{EDVR 中的 DCN}

    \item StyleGAN 中的特定的算子:\href{https://github.com/XPixelGroup/BasicSR/tree/master/basicsr/ops}{upfirdn2d, fused\_act}
\end{itemize}

如果需要使用 PyTorch C++ 扩展,可参考以下两种方式中的任意一种:

\begin{enumerate}
    \item \textbf{安装}的时候对 PyTorch C++ 扩展进行编译;
    \item 或者\textbf{每次在跑程序}的时候\textbf{即时加载 (JIT)} PyTorch C++ 扩展。
\end{enumerate}


这两种不同选择的对比如表\ref{tab:env}所示:

\begin{table}[h]
\centering
\footnotesize
\begin{tabular}{|c|c|c|c|c|}
%   \hline
%   \multicolumn{5}{|c|}{BASICSR\_EXT和BASICSR\_JIT环境变量的对比} \\
  \hline
  选项                     & 优点                  & 缺点                            & 适用场景         & 具体安装指令 \\
  \hline
  编译C++扩展 & \makecell[c]{运行代码的时候, \\ 能够快速加载扩展} & \makecell*[c]{配置环境的时候, \\ 需要更多的依赖,\\碰到的问题可能更多} & \makecell[c]{需要多次训练 \\ 或者测试模型} & \makecell[c]{在安装的时候,设置 \\BASICSR\_EXT=True}\\
  \hline
  即时加载C++扩展 & \makecell[c]{有着更少的依赖,\\碰到的问题可能更少} & \makecell[c]{每次运行代码的时候,\\都需要花费几分钟\\重新加载扩展} & 仅仅是进行测试 & \makecell[c]{在跑程序的时候,设置 \\BASICSR\_JIT=True} \\
  \hline
\end{tabular}
\caption{\label{tab:env}BASICSR\_EXT和BASICSR\_JIT环境变量的对比.}
\end{table}


\begin{note} % ---------------- Note block ---------------- %
	\textbf{注意}

	\begin{enumerate}
	    \item 对于需要在安装的时候就编译 PyTorch C++ 扩展,需要确保:gcc 和 g++ 版本 >= 5。
	    \item BasicSR\_JIT 有最高的优先级。即使在安装的时候已经成功编译了 C++ 扩展,如果在运行代码指令中设置了 BasicSR\_JIT=True,此时代码会即时加载 C++ 扩展。
	    \item 在\textbf{安装}的时候,不能设置 BasicSR\_JIT=True。
	\end{enumerate}
\end{note}


\section{BasicSR 安装}


\begin{hl} % ---------------- Highlight block ---------------- %
	\textbf{BasicSR 安装选项}

	根据不同的需求,我们提供了两种不同的安装 BasicSR 的方式。
	\begin{itemize}
		\item 如果希望去\textbf{探究 BasicSR 中的细节}或者需要对其进行\textbf{修改},推荐从本地 clone 进行安装。
		\item 如果仅仅是将 BasicSR 作为一个\textbf{功能包}进行使用 (如\href{https://github.com/TencentARC/GFPGAN}{GFPGAN},\href{https://github.com/xinntao/Real-ESRGAN}{Real-ESRGAN}),推荐直接从 PyPI 安装 BasicSR,这样可以使得自身项目的代码结构更加简洁。
	\end{itemize}
\end{hl}

\subsection{本地 clone}

要通过本地 clone 安装 BasicSR,需要依次进行以下3个步骤。

\begin{enumerate}
    \item 克隆项目:
    \begin{minted}[xleftmargin=20pt,bgcolor=bg]{python}
git clone https://github.com/XPixelGroup/BasicSR.git
    \end{minted}

    \item 安装依赖包:
    \begin{minted}[xleftmargin=20pt,bgcolor=bg]{python}
cd BasicSR
pip install -r requirements.txt
    \end{minted}

    \item 安装 BasicSR (我们提供了以下三种安装选项)

    \begin{enumerate}
    \item 如果不需要使用 PyTorch C++ 扩展:(参考\ref{section:env})
    \begin{minted}[xleftmargin=20pt,bgcolor=bg]{python}

python setup.py develop

    \end{minted}

    \item 如果仅仅希望即时加载 PyTorch C++ 扩展 :
    \begin{minted}[xleftmargin=20pt,bgcolor=bg]{python}

python setup.py develop

    \end{minted}
    \item 如果在安装的时候希望编译 PyTorch C++ 扩展,此时需要设置环境变量:
    \begin{minted}[xleftmargin=20pt,bgcolor=bg]{python}

BASICSR_EXT=True python setup.py develop

    \end{minted}
\end{enumerate}


\end{enumerate}

\label{section:clone}



如果希望安装的时候指定 CUDA 路径,可以输入如下指令:

\begin{minted}[xleftmargin=20pt,bgcolor=bg]{python}

CUDA_HOME=/usr/local/cuda \
CUDNN_INCLUDE_DIR=/usr/local/cuda \
CUDNN_LIB_DIR=/usr/local/cuda \
(BASICSR_EXT=True) python setup.py develop

\end{minted}

\subsection{pip 安装}
对于使用 pip 安装 BasicSR,我们提供了以下三种安装选项。

\begin{enumerate}
    \item 如果项目中不涉及 PyTorch C++ 扩展:
    \begin{minted}[xleftmargin=20pt,bgcolor=bg]{python}

pip install basicsr

    \end{minted}

    \item 如果仅仅希望即时加载 PyTorch C++ 扩展: (参考\ref{section:env})
    \begin{minted}[xleftmargin=20pt,bgcolor=bg]{python}

pip install basicsr

    \end{minted}

    \item 如果在安装的时候希望编译 PyTorch C++ 扩展,此时需要设置环境变量:

    \begin{minted}[xleftmargin=20pt,bgcolor=bg]{python}

BASICSR_EXT=True pip install basicsr

    \end{minted}

    \begin{note} % ---------------- Note block ---------------- %
    	\textbf{注意}

    	如果安装失败,报错信息为 \textit{ImportError: cannot import name 'deform\_conv\_ext', 'fused\_act\_ext', 'upfirdn2d\_ext'},此时需要重新安装,并输入以下指令从而在安装的时候获取详尽的日志:

    	\begin{minted}[xleftmargin=20pt,bgcolor=bg]{python}

BASICSR_EXT=True pip install basicsr -vvv

        \end{minted}

    \end{note}

\end{enumerate}


如果希望安装的时候指定 CUDA 路径,可以输入如下指令:

\begin{minted}[xleftmargin=20pt,bgcolor=bg]{python}

CUDA_HOME=/usr/local/cuda \
CUDNN_INCLUDE_DIR=/usr/local/cuda \
CUDNN_LIB_DIR=/usr/local/cuda \
(BASICSR_EXT=True) pip install basicsr

\end{minted}




\section{常见问题}


\begin{enumerate}
    \item \textbf{Q1: Windows 下是否可以使用?}

    经过验证,Windows 下可以通过上述的两种安装方式安装 BasicSR 的 CPU 版本。如果需要使用 CUDA,需要指定 CUDA 路径。由于 BasicSR 项目是在 Linux 环境下进行开发的,因此推荐在 Linux 环境下基于 BasicSR 进行项目的开发。

    \item \textbf{Q2: BASICSR\_EXT 和 BASICSR\_JIT 在什么环境下才能执行?}

    如果在加入 BASICSR\_EXT 和 BASICSR\_JIT 环境变量之后运行报错,此时需要检测 gcc 版本。BasicSR 在已被验证在 gcc5 $\sim$ gcc7 版本下可以成功编译 C++ 扩展。

    \item \textbf{Q3: 是否可以同时使用上面提到的两种方式 (clone 和 pip) 安装了 BasicSR?}

    上述两种安装方式只能选择一种进行安装,安装两个会存在路径混淆问题,并有可能会出现问题如\ref{fig:install}所示

\begin{figure}[H]
	%\vspace{-0.5cm}
	\begin{center}
		%\fbox{\rule{0pt}{2.5in} \rule{0.9\linewidth}{0pt}}
		\includegraphics[width=0.8\linewidth]{figures/section2_install.jpg}
		\caption{安装了多个 BasicSR 导致冲突}
		\label{fig:install}
	\end{center}
	\vspace{-0.5cm}
\end{figure}
    此时需要执行指令:
\begin{minted}[xleftmargin=20pt,bgcolor=bg]{python}

pip uninstall basicsr

\end{minted}

    先将安装的 BasicSR 进行卸载,随后再根据项目的需要重新选择一种方式安装 BasicSR 。

    \item \textbf{Q4: 如何确定当前是使用了哪种方式安装 BasicSR ?}

可以在命令终端输入下面的指令进行查看:
\begin{minted}[xleftmargin=20pt,bgcolor=bg]{python}

pip list

\end{minted}

    如果 basicsr 对应的地址指向的是本地的 BasicSR 项目,那就说明是通过 clone 进行安装。

    \item \textbf{Q5: 如何更新最新版本的 BasicSR?}

\begin{enumerate}
    \item 对于 clone 安装,需要将本地的 BasicSR 项目代码与\href{https://github.com/XPixelGroup/BasicSR}{远端的 BasicSR 项目代码}进行同步。
    \item 对于 pip 安装,需要先卸载当前版本的 BasicSR,再重新使用 pip 安装最新版本的 BasicSR。
\end{enumerate}

    % \item \textbf{Q6: 如何解决运行代码时出现的 version 问题?}


\end{enumerate}


\end{document}