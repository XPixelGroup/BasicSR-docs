\documentclass[../main.tex]{subfiles}

\begin{document}

\chapter{入门}
\vspace{-2cm}

本部分为 BasicSR 方法的入门部分,主要涉及的有目录解读,核心的工程文件以及其内部运行的逻辑关系,训练,测试和快速推理的流程。这个部分的主要介绍目的是希望读者能够快速的掌握BasicSR的整体结构用于自己开发使用。

\section{目录解读}\label{getting_start:content-overview}
所谓“看书先看目录”。我们首先来看一下 BasicSR 仓库的基本结构,先来整体地把握一下。根据仓库的目录层级,第一部分为仓库的整体概览。这部分主要包括算法核心文件和代码基础配置文件。具体的目录结构如下。

其中,\newline
\noindent\textcolor{red}{红色} 表示和跑实验直接相关的文件,即我们平时打交道最多的文件;\newline
\noindent\textcolor{blue}{蓝色} 表示其他与 BasicSR 强相关的代码文件;\newline
\noindent\textcolor{black}{黑色} 表示配置文件。

\vspace{0.5cm}
\renewcommand*\DTstyle{\ttfamily\textcolor{black}}
\dirtree{%
    .1 \textcolor{orange}{BasicSR 根目录}.
    .2 .github/workflows\DTcomment{GitHub 的自动 workflows,比如 PyLint、PyPI Publish等}.
    .2 .vscode\DTcomment{VSCode 配置,用于统一格式}.
    .2 LICENSE\DTcomment{使用的其他代码的 LICENSE 和 Acknowledgement}.
    .2 assets\DTcomment{存放仓库中展示使用的图片}.
    .2 \uline{\textcolor{red}{basicsr}}\DTcomment{\textcolor{red}{BasicSR 核心代码}}.
    .2 \textcolor{blue}{colab}\DTcomment{\textcolor{blue}{Google Colab 的 Notebook, 提供方便的 inference demo}}.
    .2 \textcolor{red}{datasets}\DTcomment{\textcolor{red}{“存放”使用的数据集,推荐 soft link,做到代码、数据的分离}}.
    .2 docs\DTcomment{使用和说明文档}.
    .2 \textcolor{red}{experiments}\DTcomment{\textcolor{red}{实验 checkpoints 保存路径}}.
    .3 \textcolor{blue}{pretrained\_models}\DTcomment{\textcolor{blue}{预训练模型存放路径}}.
    .2 \textcolor{red}{inference}\DTcomment{\textcolor{red}{快速推理,主要用于得到 demo 结果}}.
    .2 \uline{\textcolor{red}{options}}\DTcomment{\textcolor{red}{训练和测试的配置文件}}.
    .2 \uline{\textcolor{red}{scripts}}\DTcomment{\textcolor{red}{功能脚本,包含数据集制作,指标测试和数据集下载等}}.
    .2 \textcolor{blue}{test\_scripts}\DTcomment{\textcolor{blue}{一些用于手动单元测试的脚本}}.
    .2 \textcolor{blue}{tests}\DTcomment{\textcolor{blue}{PyTest 自动单元测试}}.
    .2 .gitignore\DTcomment{Git 忽略文件的配置}.
    .2 .pre-commit-config.yaml\DTcomment{Pre-commit Hook 的配置文件}.
    .2 .readthedocs.yaml\DTcomment{自动触发 basicsr.readthedocs.io 的配置文件}.
    .2 MANIFEST.in\DTcomment{发布 basicsr 时,额外需要包含进去的文件}.
    .2 README.md\DTcomment{说明文档}.
    .2 README\_CN.md\DTcomment{说明文档中文版}.
    .2 \textcolor{blue}{VERSION}\DTcomment{\textcolor{blue}{版本文件}}.
    .2 \textcolor{blue}{requirements.txt}\DTcomment{ \textcolor{blue}{安装依赖包文件}}.
    .2 setup.cfg\DTcomment{格式配置文件,比如 flake8,yapf 和 isort}.
    .2 \textcolor{blue}{setup.py}\DTcomment{\textcolor{blue}{安装文件}}.
}

\vspace{1cm}

在 BasicSR 仓库中,核心代码在 basicsr 这个文件夹中。这个部分主要为深度学习模型常用的代码文件,比如网络结构,损失函数和数据加载等,具体目录如下:
\dirtree{%
    .1 \textcolor{red}{basicsr}.
    .2 \textcolor{red}{archs}\DTcomment{定义网络结构}.
    .2 \textcolor{red}{data}\DTcomment{定义训练数据(Dataloader)}.
    .2 \textcolor{red}{losses}\DTcomment{定义损失函数}.
    .2 metrics\DTcomment{定义评价指标}.
    .2 \textcolor{red}{models}\DTcomment{定义训练过程(前向,反向传播,梯度优化,测试等)}.
    .2 ops\DTcomment{定义训练中的组件,例如styleGAN的算子}.
    .2 utils\DTcomment{定义基础工具,例如file client和registry等}.
    .2 \textcolor{red}{test.py}\DTcomment{定义测试流程}.
    .2 \textcolor{red}{train.py}\DTcomment{定义训练流程}.
}

由于在算法设计和开发中数据的预处理和测试等是必备的操作,相关的文件位于scripts,目录如下:
\dirtree{%
    .1 \textcolor{red}{scripts}.
    .2 \textcolor{red}{data\_preparation}\DTcomment{准备数据}.
    .2 matlab\_scripts\DTcomment{MATLAB语言的数据处理脚本}.
    .2 metrics\DTcomment{评价指标,例如PSNR,SSIM和NIQE等}.
    .2 model\_conversion\DTcomment{xx.pth模型的key转换脚本}.
    .2 dist\_test.sh\DTcomment{分布式测试启动脚本}.
    .2 dist\_train.sh\DTcomment{分布式训练启动脚本}.
    .2 download\_gdrive.py\DTcomment{从Google Drive下载文件}.
    .2 download\_pretrained\_models.py\DTcomment{从Google Drive批量下载预训练模型}.
    .2 publish\_models\DTcomment{定义测试流程}.
}


\section{训练流程}

在对目录结构有了初步的了解之后就可以进行训练了,本节主要是对训练流程对一些主要步骤进行介绍,比如训练对命令行以及相关参数和训练的主文件的代码块的功能逻辑。本节的目的是希望能够初步的让读者了解到训练的基本流程和代码逻辑流,具体的细节我们会采用引用的方式来供读者查阅。


\begin{enumerate}
    \item 首先需要在终端输入命令来开始训练:

\begin{minted}[xleftmargin=20pt,breaklines,bgcolor=bg]{bash}
python basicsr/train.py -opt options/train/SRResNet_SRGAN/train_MSRResNet_x4.yml
\end{minted}

    其中train\_MSRResNet\_x4为yml配置文件,主要设置实验相关的超参数和相关的一些配置参数。

    对于一些大型模型,可能需要采用分布式来进行训练,相关的命令可参考:
    \href{https://github.com/XPixelGroup/BasicSR/blob/master/docs/TrainTest.md}{分布式训练和测试命令}

    \item 在训练的命令行中,可以采用一些参数来进行训练对配置

    \begin{enumerate}
    \item \textbf{$-$opt}:配置文件的路径。
    \item \textbf{$--$laucher}:用于指定分布式训练,比如pytorch或者slurm。默认是none,即单卡非distributed training。
    \item \textbf{$--$auto\_resume}:自动查找最近的checkpoint继续训练。程序异常中断,在命令行中添加这个参数,就可以实现自动resume。
    \item \textbf{$--$debug}:能够快速帮助debug。进入debug模式后,每个iter都会print log,8个iter后就会validation。这样可以快速方便地查看代码是否可以正常运行。
    \item \textbf{$--$local\_rank}:分布式训练程序自动会传入
    \item \textbf{$--$force\_yml}:在命令行中修改yml中的配置文件。一般不推荐使用,除非你非常清楚自己所做的事。
    \end{enumerate}

\item 训练文件

输入指令后,BasicSR/basicsr/train.py文件中的代码就开始执行训练
\begin{minted}[xleftmargin=20pt,linenos,bgcolor=bg]{python}
if __name__ == '__main__':
    root_path = osp.abspath(osp.join(__file__, osp.pardir, osp.pardir))
    train_pipeline(root_path)
\end{minted}
首先需要把root\_path作为参数传进去,当我们把basicsr作为package使用的时候,需要根据当前的目录路径来创建文件,否则程序会错误地使用basicsr package所在位置的目录了。

其中,train\_pipeline函数的流程主要为主要有:
\begin{enumerate}

\item 初始化定义

\begin{minted}[xleftmargin=20pt,linenos,breaklines,bgcolor=bg]{python}
def train_pipeline(root_path):
    opt, args = parse_options(root_path, is_train=True)
    opt['root_path'] = root_path
# 解析yml配置文件和其他配置操作,比如分布式训练和随机种子等
    torch.backends.cudnn.benchmark = True
    resume_state = load_resume_state(opt)
# 如果有resume指令,则会加载设置的.stat文件
    if resume_state is None:
        make_exp_dirs(opt)
        if opt['logger'].get('use_tb_logger') and 'debug'
        not in opt['name'] and opt['rank'] == 0:
            mkdir_and_rename(osp.join(opt['root_path'],
            'tb_logger', opt['name']))
# 创建实验所需文件夹,如果有resume则不需要创建
    copy_opt_file(args.opt, opt['path']['experiments_root'])
# 注意在这以上代码(及函数调用)中,不能使用get\_root\_logger,
# 否则会导致logger初始化错误
    log_file = osp.join(opt['path']['log'],
    f"train_{opt['name']}_{get_time_str()}.log")
    logger = get_root_logger(logger_name='basicsr', log_level=logging.INFO,
    log_file=log_file)
    logger.info(get_env_info())
# 初始化日志系统logger,生成文件和屏幕的logger显示
    logger.info(dict2str(opt))
# 输出环境和配置信息
    tb_logger = init_tb_loggers(opt)
# 根据配置需要,初始化tensorboard和wandb的logger

\end{minted}

\item 数据和模型等的初始化

\begin{minted}[xleftmargin=20pt,linenos,breaklines,bgcolor=bg]{python}
# 创建训练和验证的dataloaders
result = create_train_val_dataloader(opt, logger)
train_loader, train_sampler, val_loaders, total_epochs, total_iters = result
# 创建模型
model = build_model(opt)
if resume_state:  # resume training
    model.resume_training(resume_state)  # handle optimizers and schedulers
    logger.info(f"Resuming training from epoch: {resume_state['epoch']},
    iter: {resume_state['iter']}.")
    start_epoch = resume_state['epoch']
    current_iter = resume_state['iter']
else:
    start_epoch = 0
    current_iter = 0
# 创建实验信息的logger来输出
msg_logger = MessageLogger(opt, current_iter, tb_logger)
# dataloader prefetcher
prefetch_mode = opt['datasets']['train'].get('prefetch_mode')
if prefetch_mode is None or prefetch_mode == 'cpu':
    prefetcher = CPUPrefetcher(train_loader)
elif prefetch_mode == 'cuda':
    prefetcher = CUDAPrefetcher(train_loader, opt)
    logger.info(f'Use {prefetch_mode} prefetch dataloader')
    if opt['datasets']['train'].get('pin_memory') is not True:
        raise ValueError('Please set pin_memory=True for CUDAPrefetcher.')
else:
    raise ValueError(f"Wrong prefetch_mode {prefetch_mode}.
    Supported ones are: None, 'cuda', 'cpu'.")

\end{minted}


\item 训练流程
本小节是训练过程中对主要代码块,在经过了训练参数的初始化,数据和模型的初始化了之后,程序就从start\_epoch开始训练只至total\_epochs+1训练结束。其中主要的功能逻辑流为:

\begin{enumerate}
\item 更新学习率
\item 加载数据
\item 执行一次训练,更新网络参数
\item 保存日志信息
\item 达到预设迭代次数,执行验证或者保存模型
\end{enumerate}

\begin{minted}[xleftmargin=20pt,linenos,breaklines,bgcolor=bg]{python}
# line149 - 200
logger.info(f'Start training from epoch: {start_epoch}, iter: {current_iter}')
data_timer, iter_timer = AvgTimer(), AvgTimer()
start_time = time.time()

for epoch in range(start_epoch, total_epochs + 1):
    train_sampler.set_epoch(epoch)
    prefetcher.reset()
    train_data = prefetcher.next()
# 以epoch为外层循环,但是在yml配置文件中使用的是iteration来进行设置
    while train_data is not None:
# 当每一个epoch还有数据,就进入一次iteration的训练
        data_timer.record()
        current_iter += 1
        if current_iter > total_iters:
            break
# 达到设置的最大训练iteration数目,训练停止
        model.update_learning_rate(current_iter,
        warmup_iter=opt['train'].get('warmup_iter', -1))
# 更新学习率
        model.feed_data(train_data)
# 加入数据
        model.optimize_parameters(current_iter)
# 执行一次训练
        iter_timer.record()
        if current_iter == 1:
            # 重新设置msg_logger的开始时间
            # resume 模式下不启用
            msg_logger.reset_start_time()
        # log 文件设置
        if current_iter % opt['logger']['print_freq'] == 0:
            log_vars = {'epoch': epoch, 'iter': current_iter}
            log_vars.update({'lrs': model.get_current_learning_rate()})
            log_vars.update({'time': iter_timer.get_avg_time(),
            'data_time': data_timer.get_avg_time()})
            log_vars.update(model.get_current_log())
            msg_logger(log_vars)
# 保存训练的实验参数,比如loss等
        if current_iter % opt['logger']['save_checkpoint_freq'] == 0:
            logger.info('Saving models and training states.')
            model.save(epoch, current_iter)
# 根据yml设置的参数,每隔一段时间保存模型(.pth)和训练状态文件(.state)
        # validation
        if opt.get('val') is not None and (current_iter
            if len(val_loaders) > 1:
                logger.warning('Multiple validation datasets are
                *only* supported by SRModel.')
            for val_loader in val_loaders:
                model.validation(val_loader, current_iter,
                tb_logger, opt['val']['save_img'])
# 根据yml设置的参数,进行一次validation
        data_timer.start()
        iter_timer.start()
        train_data = prefetcher.next()
    # 结束一个iteration
# 结束一个epoch
\end{minted}

\end{enumerate}

% 4. 相关的函数定义

% \begin{enumerate}
% \item build\_dataset和build\_dataloader
% \item build\_model
% \end{enumerate}

\end{enumerate}


\section{测试流程}


测试阶段,我们需要在终端输入命令来开始训练。
\begin{minted}[xleftmargin=20pt,linenos,breaklines,bgcolor=bg]{python}
CUDA_VISIBLE_DEVICES=0 python basicsr/test.py -opt options/test/SRResNet_SRGAN/test_MSRResNet_x4.yml
\end{minted}

其中test\_MSRResNet\_x4为yml配置文件,主要设置实验相关的超参数和相关的一些配置参数。

\begin{hl} % ---------------- Highlight block ---------------- %

    \textbf{测试注意事项}

    如果需要额外添加测试集,需要使用test\_x格式,下划线之后是测试集的编号

    路径需要设置为待测模型所在的路径

    suffix配置为存储图片的名字后缀,一般设置为这个方法的名字便于对比

    metrics里面设置需要的评价指标,PSNR,SSIM和NIQE等

    分布式测试命令可参考:
\href{https://github.com/XPixelGroup/BasicSR/blob/master/docs/TrainTest.md}{分布式训练和测试命令}

\end{hl}





\begin{minted}[xleftmargin=20pt,linenos,bgcolor=bg]{python}

def test_pipeline(root_path):
    # 加载配置参数
    opt, _ = parse_options(root_path, is_train=False)
    torch.backends.cudnn.benchmark = True

    # 新建loggers并初始化
    make_exp_dirs(opt)
    log_file = osp.join(opt['path']['log'],
    f"test_{opt['name']}_{get_time_str()}.log")
    logger = get_root_logger(logger_name='basicsr',
    log_level=logging.INFO, log_file=log_file)
    logger.info(get_env_info())
    logger.info(dict2str(opt))

    # 创建测试集和dataloader
    test_loaders = []
    for _, dataset_opt in sorted(opt['datasets'].items()):
        test_set = build_dataset(dataset_opt)
        test_loader = build_dataloader(
            test_set, dataset_opt, num_gpu=opt['num_gpu'],
            dist=opt['dist'], sampler=None, seed=opt['manual_seed'])
        logger.info(f"Number of test images in {dataset_opt['name']}:
        {len(test_set)}")
        test_loaders.append(test_loader)

    # 创建模型
    model = build_model(opt)
    # 测试多个测试集
    for test_loader in test_loaders:
        test_set_name = test_loader.dataset.opt['name']
        logger.info(f'Testing {test_set_name}...')
        model.validation(test_loader, current_iter=opt['name'],
        tb_logger=None, save_img=opt['val']['save_img'])


if __name__ == '__main__':
    root_path = osp.abspath(osp.join(__file__, osp.pardir, osp.pardir))
    test_pipeline(root_path)
\end{minted}


\section{推理流程}

快速推理阶段,我们需要在终端输入命令来进行快速推理:
\begin{minted}[xleftmargin=20pt,linenos,breaklines,bgcolor=bg]{python}
CUDA_VISIBLE_DEVICES=0 python basicsr/inference/inference_esrgan.py --input input_path --output out_path
\end{minted}

\begin{hl} % ---------------- Highlight block ---------------- %

    \textbf{推理注意事项}

    需要提前下载好预训练模型放在experiments/pretrained\_models/ 下面,在配置参数中可以详细的对比下载存放的路径和代码中设置的路径

    和测试不同,快速推理不涉及到定量评价指标的计算,比如PSNR,SSIM和NIQE等

    分布式测试命令可参考:
\href{https://github.com/XPixelGroup/BasicSR/blob/master/docs/TrainTest.md}{分布式训练和测试命令}

\end{hl}



\begin{minted}[xleftmargin=20pt,linenos,breaklines,bgcolor=bg]{python}

#加载配置参数
parser = argparse.ArgumentParser()
parser.add_argument(
    '--model_path',
    type=str,
    default=  # noqa: E251
    'experiments/pretrained_models/ESRGAN/
    ESRGAN_SRx4_DF2KOST_official-ff704c30.pth'  # noqa: E501
)
parser.add_argument('--input', type=str,
default='datasets/Set14/LRbicx4', help='input test image folder')
parser.add_argument('--output', type=str, default='results/ESRGAN',
help='output folder')
args = parser.parse_args()

device = torch.device('cuda' if torch.cuda.is_available() else 'cpu')
# 设置快速推理所需要模型
model = RRDBNet(num_in_ch=3, num_out_ch=3, num_feat=64,
num_block=23, num_grow_ch=32)
model.load_state_dict(torch.load(args.model_path)['params'], strict=True)
model.eval()
model = model.to(device)

# 设置快速推理后输出的存放路径
os.makedirs(args.output, exist_ok=True)
for idx, path in enumerate(sorted(glob.glob(os.path.join(args.input, '*')))):
    imgname = os.path.splitext(os.path.basename(path))[0]
    print('Testing', idx, imgname)
    # 读取图片
    img = cv2.imread(path, cv2.IMREAD_COLOR).astype(np.float32) / 255.
    img = torch.from_numpy(np.transpose(img[:, :, [2, 1, 0]], (2, 0, 1))).float()
    img = img.unsqueeze(0).to(device)
    # 快速推理
    try:
        with torch.no_grad():
            output = model(img)
    except Exception as error:
        print('Error', error, imgname)
    else:
        # 保存图片
        output = output.data.squeeze().float().cpu().clamp_(0, 1).numpy()
        output = np.transpose(output[[2, 1, 0], :, :], (1, 2, 0))
        output = (output * 255.0).round().astype(np.uint8)
        cv2.imwrite(os.path.join(args.output, f'{imgname}_ESRGAN.png'), output)
\end{minted}

\section{入门样例}

这个部分我们以一个基础的超分模型SRResNet作为例子来展示BasicSR的入门使用。相关的文件目录如下所示:

\dirtree{%
    .1 \textcolor{black}{BasicSR}.
    .2 \textcolor{red}{basicsr}\DTcomment{BasicSR核心代码包}.
    .2 scripts\DTcomment{常用脚本}.
    .3 data\_preparation\DTcomment{数据准备脚本目录}.
    .4 extract\_subimages.py\DTcomment{生成子图脚本}.
    .2 datasets\DTcomment{数据集存放,推荐soft link}.
    .3 DIV2K\DTcomment{训练数据集}.
    .4 DIV2K\_train\_HR\_sub\DTcomment{训练数据集的GT子图}.
    .4 DIV2K\_train\_LR\_bicubic\_X4\_sub\DTcomment{训练数据集子图的下采样图}.
    .3 Set5\DTcomment{验证集}.
    .4 GTmod12\DTcomment{验证集的GT图}.
    .4 LRbicx4\DTcomment{验证集的下采样图}.
    .3 Set14\DTcomment{验证集}.
    .4 GTmod12\DTcomment{验证集的GT图}.
    .4 LRbicx4\DTcomment{验证集的下采样图}.
    .2 experiments\DTcomment{实验保存路径}.
    .3 pretrained\_models\DTcomment{预训练模型保存路径}.
    .3 001\_MSRResNet\_x4\_f64b16\_DIV2K\_1000k\_B16G1\_wandb\DTcomment{SRResNet实验存放路径}.
    .4 models\DTcomment{SRResNet训练模型存放位置}.
    .4 visualization\DTcomment{SRResNet实验验证图像}.
    .4 training\_states\DTcomment{SRResNet实验resume文件存放路径}.
    .4 001\_MSRResNet\_x4\_f64b16\_DIV2K\_1000k\_B16G1\_wandb\DTcomment{SRResNet实验配置文件}.
    .2 inference\DTcomment{快速推理获得结果}.
    .2 options\DTcomment{训练和测试配置文件}.
    .3 train\DTcomment{训练配置文件夹}.
    .4 SRResNet\_SRGAN\DTcomment{SRResNet训练配置文件夹}.
    .5 train\_MSRResNet\_x4.yml\DTcomment{SRResNet训练配置文件}.
    .3 test\DTcomment{测试配置文件夹}.
    .4 SRResNet\_SRGAN\DTcomment{SRResNet测试配置文件夹}.
    .5 test\_MSRResNet\_x4.yml\DTcomment{SRResNet测试配置文件}.
    % .4 train\_MSRResNet_x4.yml\DTcomment{SRResNet配置文件}.
    % .2 \textcolor{red}{scripts}\DTcomment{功能脚本,包含数据集制作,指标测试和数据集下载等}.
}

\begin{enumerate}

\item 第一步是下载训练所用的数据集,常用的数据集链接可以参考:
\href{https://github.com/XPixelGroup/BasicSR/blob/master/docs/DatasetPreparation.md#DIV2K}{https://github.com/XPixelGroup/BasicSR/blob/master/docs/DatasetPreparation.md\#DIV2K}

在这里我们采用DIV2K 作为训练数据集,Set5作为验证集

\begin{exampleBox}[]{数据集链接}

DIV2K:
\href{https://data.vision.ee.ethz.ch/cvl/DIV2K/}{https://data.vision.ee.ethz.ch/cvl/DIV2K/}

Set5和Set14:
\href{https://drive.google.com/drive/folders/1B3DJGQKB6eNdwuQIhdskA64qUuVKLZ9u}{https://drive.google.com/drive/folders/1B3DJGQKB6eNdwuQIhdskA64qUuVKLZ9u}

\end{exampleBox}

将下载好的训练集和验证集放在datasets目录下。(软链接是更好的方式,这里为了进行入门样例展示,采用了直接存放数据集的方式)

\item 第二步是将下载好的DIV2K数据集切成子图的形式存放在DIV2K\_train\_HR\_sub目录下,由于2K图像的读取会占用大量的时间所以采用子图的形式进行读入

\href{https://github.com/XPixelGroup/BasicSR/blob/master/scripts/data_preparation/extract_subimages.py}{https://github.com/XPixelGroup/BasicSR/blob/master/scripts/data\_preparation/extract\_subimages.py}

\begin{minted}[xleftmargin=20pt,linenos,breaklines,bgcolor=bg]{python}

    # HR images 这个过程将2K的图像给切成480X480的子图
    # 原始2K图像路径
    opt['input_folder'] = 'datasets/DIV2K/DIV2K_train_HR'
    # 子图存放路径
    opt['save_folder'] = 'datasets/DIV2K/DIV2K_train_HR_sub'
    opt['crop_size'] = 480 # 子图的尺寸
    opt['step'] = 240 # 切图的步长
    opt['thresh_size'] = 0
    extract_subimages(opt)

    # LRx4 images
    opt['input_folder'] = 'datasets/DIV2K/DIV2K_train_LR_bicubic/X4'
    opt['save_folder'] = 'datasets/DIV2K/DIV2K_train_LR_bicubic/X4_sub'
    opt['crop_size'] = 120 # 子图的尺寸
    opt['step'] = 60 # 切图的步长

\end{minted}

\item 制作好数据之后,修改yml配置文件中训练集和验证集的路径,就可以初步的把实验配置完成了

% \begin{exampleBox}[]{配置文件修改}

%     dataroot_gt: datasets/DF2K/DIV2K_train_HR_sub
%     dataroot_lq: datasets/DF2K/DIV2K_train_LR_bicubic_X4_sub

% \end{exampleBox}

\begin{minted}[xleftmargin=20pt,linenos,breaklines,bgcolor=bg]{python}

# dataroot_gt: datasets/DF2K/DIV2K_train_HR_sub
# dataroot_lq: datasets/DF2K/DIV2K_train_LR_bicubic_X4_sub
# DF2K是DIV2K和Flickr2K数据集合并的数据集,这里我们先用DIV2K进行实验,可以根据需求调整自己的数据集
dataroot_gt: datasets/DIV2K/DIV2K_train_HR_sub
dataroot_lq: datasets/DIV2K/DIV2K_train_LR_bicubic_X4_sub

\end{minted}

\item 训练命令和log显示

\begin{minted}[xleftmargin=20pt,linenos,breaklines,bgcolor=bg]{python}
python basicsr/train.py -opt options/train/SRResNet_SRGAN/train_MSRResNet_x4.yml
\end{minted}

执行训练命令之后,终端会打印出训练的相关信息和验证集上的PSNR的精度,如下所示:

\begin{minted}[xleftmargin=20pt,linenos,breaklines,bgcolor=bg]{python}

2020-08-21 00:13:08,623 INFO: [001_M..][epoch:  4, iter:1,000,000, lr:(1.000e-07,)] [eta: 0:00:00, time (data): 0.041 (0.000)] l_pix: 2.1622e-02
2020-08-21 00:13:08,624 INFO: Saving models and training states.
... ...
Test head
[>>>>>>>>>>>>>>>>>>>>>>>>>>>>>>>>>>>>>>>>>>>>>>>>>>] 5/5, 21.3 task/s, elapsed: 0s, ETA:     0s
Test woman
2020-08-21 00:13:08,926 INFO: Validation Set5
 # psnr: 30.2497

\end{minted}

同时,我们配置了wandb之后,可以在自己的wandb云端主页上看到训练曲线,wandb的配置见xx
\begin{figure}[h]
    \vspace{1cm}
    \begin{center}
        %\fbox{\rule{0pt}{2.5in} \rule{0.9\linewidth}{0pt}}
        \includegraphics[width=\linewidth]{figures/SRResNet_psnr_curve.jpg}
        %\vspace{-1cm}
        \caption{SRResNet的PSNR训练曲线}
        %\label{fig:logo}
    \end{center}
    %\vspace{-0.7cm}
\end{figure}

5. 测试过程,我们采用以下命令来对训练好的模型进行测试
\begin{minted}[xleftmargin=20pt,linenos,breaklines,bgcolor=bg]{python}
CUDA_VISIBLE_DEVICES=0 python basicsr/test.py -opt options/test/SRResNet_SRGAN/test_MSRResNet_x4.yml
\end{minted}

其中,我们需要准备好待测数据集,比如test\_MSRResNet\_x4.yml配置文件中的DIV2K100测试集,需要我们按照第一步数据集准备的过程进行制作,然后设置对应路径

\begin{minted}[xleftmargin=20pt,linenos,breaklines,bgcolor=bg]{python}
  test_3:
    name: DIV2K100
    type: PairedImageDataset
    dataroot_gt: datasets/DIV2K/DIV2K_valid_HR
    dataroot_lq: datasets/DIV2K/DIV2K_valid_LR_bicubic/X4
\end{minted}

此外,我们还需要对模型存放的路径进行设置
\begin{minted}[xleftmargin=20pt,linenos,breaklines,bgcolor=bg]{python}
path:
  pretrain_network_g: experiments/001_MSRResNet_x4_f64b16_DIV2K_1000k_B16G1_wandb/
  models/net_g_1000000.pth
\end{minted}

设置完成后执行测试命令就可以在results文件夹下得到测试的结果图和PSNR等定量的指标结果。

\end{enumerate}

\end{document}