\documentclass[../main.tex]{subfiles}

\begin{document}

\chapter{如何添加与修改}
\vspace{-2cm}

本章主要介绍如何在 BasicSR 框架中添加自定义的 Dataset ,网络结构 (Architecture),模型 (Model),损失函数 (Loss) 以及指标 (Metric)。
使用者需要关注四个方面,即:
\begin{enumerate}
    \item 相关文件的存放和命名
    \item 编写自定义文件
    \item 注册新添加类
    \item 以及在配置文件中进行设置
\end{enumerate}

这一部分的内容大体上十分相似,使用者只要对某一个模块比较熟悉(如添加修改网络结构),即可快速类比至其他各个部分。
在添加新的自定义模块时,理解并参考已有文件可以帮助使用者快速上手。

值得提及的是,当用户使用 \textbf{BasicSR-template} 进行开发,尤其是针对\textbf{指标}模块时,以下操作可能并不完全适用,具体详见第\ref{chapter:template}章:\nameref{chapter:template} 相关部分。

% ------------------------------------------------------------------------------
\section{添加修改 Dataset}\label{howto:add_dataset}
\begin{enumerate}[第 1 步]
    \item Dataset 文件的存放与命名:Dataset 文件存放在 \href{https://github.com/XPixelGroup/BasicSR/tree/master/basicsr/data}{basicsr/data/} 文件夹下。例如,\href{https://github.com/XPixelGroup/BasicSR/tree/master/basicsr/data}{basicsr/data/paired\_image\_dataset.py}。用户可根据需求对已有的 Dataset 进行修改,或是添加自定义 Dataset 文件。在创建新的自定义 Dataset 文件时,注意文件名需以  \textbf{\_dataset.py} 作为结尾

    \item 编写自定义 Dataset :在 Dataset 文件中对自定义 Dataset 类进行命名,\textbf{需要注意新建类名不能与已有类名重复,否则会导致后续注册机制报错}。关于 Dataset 文件中函数功能详解见章节\ref{code_structure:data}:\nameref{code_structure:data},此处不再赘述。对于需要添加新的设置参数,用户可以灵活利用 \textbf{opt} 参数从配置文件中读取

    \item 注册 Dataset :用户需要对新建的 Dataset 类进行注册。注册机制的原理详见章节\ref{code_structure:register}:\nameref{code_structure:register}。此处具体操作为,首先对 \textbf{DATASET\_REGISTRY} 函数进行导入,然后在新建类上方添加修饰器来注册新建函数。以 \href{https://github.com/XPixelGroup/BasicSR/tree/master/basicsr/data/paired_image_dataset.py}{paired\_image\_dataset.py} 中的 \textbf{PairedImageDataset} 为例:

          \begin{minted}[xleftmargin=20pt,linenos,bgcolor=bg]{python}
from basicsr.utils.registry import DATASET_REGISTRY

@DATASET_REGISTRY.register()
class PairedImageDataset(data.Dataset):
    ...
\end{minted}

    \item 在配置文件中设置自定义 Dataset :将配置文件(即 YAML 文件)中 \textbf{datasets} 部分中 \textbf{type} 参数设置为新建的 Dataset 类名即可。
          该部分其余参数的功能与 Dataset 中用户自定义的功能对应。以使用 \href{https://github.com/XPixelGroup/BasicSR/tree/master/basicsr/data/paired_image_dataset.py}{paired\_image\_dataset.py} 中的 \textbf{PairedImageDataset} 为例:

          \begin{minted}[xleftmargin=20pt,linenos,bgcolor=bg]{python}
# dataset and data loader settings
dataset:
  ...
  type: PairedImageDataset  # 设置为需要使用的 Dataset 类名
  ...
\end{minted}
\end{enumerate}

% ------------------------------------------------------------------------------
\section{添加修改模型}\label{howto:add_model}

\begin{enumerate}[第 1 步]
    \item 模型文件的存放与命名:模型文件存放在 \href{https://github.com/XPixelGroup/BasicSR/tree/master/basicsr/models}{basicsr/models/} 文件夹下。例如,\href{https://github.com/XPixelGroup/BasicSR/tree/master/basicsr/models}{basicsr/archs/sr\_model.py}。用户可根据需求对已有的模型进行修改,或是添加自定义模型文件。在创建新的自定义模型文件时,注意文件名需以  \textbf{\_model.py} 作为结尾。

    \item 编写自定义模型:在模型文件中对自定义模型类进行命名,\textbf{需要注意新建类名不能与已有类名重复,否则会导致后续注册机制报错}。关于模型文件中的函数功能详解见章节\ref{code_structure:model}:\nameref{code_structure:model}。模型部分涉及的函数较多,但一般情况下需要改写的部分非常有限。用户往往只需要继承已有模型,并对需要更改的函数进行重构即可。以 \href{https://github.com/XPixelGroup/BasicSR/tree/master/basicsr/models}{basicsr/archs/swinir\_model.py} 中的 \textbf{SwinIRModel} 为例,该模型相较于图像超分通用的 \href{https://github.com/XPixelGroup/BasicSR/tree/master/basicsr/models}{basicsr/archs/sr\_model.py} 中的 \textbf{SRModel} 仅需更改 test 函数,因此 \textbf{SwinIRModel} 类在继承了 \textbf{SRModel} 的基础上只对 \textbf{test} 函数进行了重构:

          \begin{minted}[xleftmargin=20pt,linenos,bgcolor=bg]{python}
class SwinIRModel(SRModel):  # SwinIRModel 继承自 SRModel
    def test(self):  # 重构 test 函数
        ...
\end{minted}

    \item 注册模型:用户需要对新建的模型类进行注册。注册机制的原理详见章节\ref{code_structure:register}:\nameref{code_structure:register}。此处具体操作为,首先对 \textbf{MODEL\_REGISTRY} 函数进行导入,然后在新建类上方添加修饰器来注册新建类。以 \href{https://github.com/XPixelGroup/BasicSR/tree/master/basicsr/models/sr_model.py}{sr\_model.py} 中的 \textbf{SRModel} 为例:

          \begin{minted}[xleftmargin=20pt,linenos,bgcolor=bg]{python}
from basicsr.utils.registry import MODEL_REGISTRY

@MODEL_REGISTRY.register()
class SRModel(nn.Module):
    ...
\end{minted}

    \item 在配置文件中设置自定义模型:将配置文件(即 YAML 文件)中 \textbf{general settings} 部分中的 \textbf{model\_type} 参数设置为新建的模型类名即可。以使用 \href{https://github.com/XPixelGroup/BasicSR/tree/master/basicsr/models/sr_model.py}{sr\_model.py} 中的 \textbf{SRModel} 为例:

          \begin{minted}[xleftmargin=20pt,linenos,bgcolor=bg]{python}
# general settings
...
type: SRModel  # 设置为需要使用的模型类名
...
\end{minted}
          除此之外,模型与整个配置文件的内容都是息息相关的,涉及到数据的读取与处理、模型网络结构、训练优化和测试评估等几乎所有内容的设置组成,而非一个独立的部分。用户在修改配置文件的结构时,建议参考已有文件作为模板,重点对模型进行修改的部分在配置文件中做对应处理。

\end{enumerate}

% ------------------------------------------------------------------------------
\section{添加修改网络结构}\label{howto:add_arch}

\begin{enumerate}[第 1 步]
    \item 网络结构文件的存放与命名:网络结构文件存放在 \href{https://github.com/XPixelGroup/BasicSR/tree/master/basicsr/archs}{basicsr/archs/} 文件夹下。例如,\href{https://github.com/XPixelGroup/BasicSR/tree/master/basicsr/archs}{basicsr/archs/srresnet\_arch.py}。用户可根据需求对已有的网络结构进行修改,或是添加自定义网络结构文件。在创建新的自定义网络结构文件时,注意文件名需以  \textbf{\_arch.py} 作为结尾。

    \item 编写自定义网络结构:在网络结构文件中对自定义网络结构类进行命名,\textbf{需要注意新建类名不能与已有类名重复,否则会导致后续注册机制报错}。关于网络结构文件中的函数功能详解见章节\ref{code_structure:arch}:\nameref{code_structure:arch}。对于需要手工设置的参数,用户可以灵活利用 \textbf{opt} 参数从配置文件中读取。

    \item 注册网络结构:用户需要对新建的网络结构类进行注册。注册机制的原理详见章节\ref{code_structure:register}:\nameref{code_structure:register}。此处具体操作为,首先对 \textbf{ARCH\_REGISTRY} 函数进行导入,然后在新建类上方添加修饰器来注册新建类。以 \href{https://github.com/XPixelGroup/BasicSR/tree/master/basicsr/archs/srresnet_arch.py}{srresnet\_arch.py} 中的 \textbf{MSRResNet} 类为例:
          \begin{minted}[xleftmargin=20pt,linenos,bgcolor=bg]{python}
from basicsr.utils.registry import ARCH_REGISTRY

@ARCH_REGISTRY.register()
class MSRResNet(nn.Module):
    ...
\end{minted}

    \item 在配置文件中设置自定义网络结构:将配置文件(即 YAML 文件)中 \textbf{network structures} 部分中的 \textbf{type} 参数设置为新建的网络结构类名即可。
          该部分其余参数的功能与模型和网络结构中用户自定义的功能对应。以使用 \href{https://github.com/XPixelGroup/BasicSR/tree/master/basicsr/archs/srresnet_arch.py}{srresnet\_arch.py} 中的 \textbf{MSRResNet} 为例:

          \begin{minted}[xleftmargin=20pt,linenos,bgcolor=bg]{python}
# network structures
network_g:  # g网络设置
  ...
  type: MSRResNet  # 设置为需要使用的网络结构类名
  ...
\end{minted}
\end{enumerate}

% ------------------------------------------------------------------------------
\section{添加修改损失函数}\label{howto:add_loss}

\begin{enumerate}[第 1 步]
    \item 损失函数的存放与命名:损失函数文件存放在 \href{https://github.com/XPixelGroup/BasicSR/tree/master/basicsr/losses}{basicsr/losses/} 文件夹下。例如,\href{https://github.com/XPixelGroup/BasicSR/blob/master/basicsr/losses/gan_loss.py}{basicsr/losses/gan\_loss.py}。用户可根据需求对已有的损失函数进行修改,或是添加自定义损失函数文件。在创建新的损失函数文件时,注意文件名需以  \textbf{\_loss.py} 作为结尾。

    \item 编写自定义损失函数:在损失函数文件中对自定义损失函数类进行命名,\textbf{需要注意新建类名不能与已有类名重复,否则会导致后续注册机制报错}。关于损失函数的功能详解见章节\ref{code_structure:loss}:\nameref{code_structure:loss}。对于需要手工设置的参数,用户可以灵活利用 \textbf{opt} 参数从配置文件中读取。

    \item 注册损失函数:用户需要对新建的损失函数类进行注册。注册机制的原理详见章节\ref{code_structure:register}:\nameref{code_structure:register}。此处具体操作为,首先对 \textbf{LOSS\_REGISTRY} 函数进行导入,然后在新建类上方添加修饰器来注册新建类。以 \href{https://github.com/XPixelGroup/BasicSR/blob/master/basicsr/losses/basic_loss.py}{basicsr/losses/basic\_loss.py} 中的 \textbf{L1Loss} 为例:
          \begin{minted}[xleftmargin=20pt,linenos,bgcolor=bg]{python}
from basicsr.utils.registry import LOSS_REGISTRY

@LOSS_REGISTRY.register()
class L1Loss(nn.Module):
    ...
\end{minted}

    \item 在配置文件中设置自定义损失函数:将配置文件(即 YAML 文件)中 \textbf{losses} 部分中相应损失函数项的 \textbf{type} 参数设置为新建的损失类名即可。需要注意损失函数项的存在与模型有关。以使用 \href{https://github.com/XPixelGroup/BasicSR/tree/master/basicsr/losses/basic_loss.py}{basicsr/losses/basic\_loss.py} 中的 \textbf{L1Loss} 为例:
          \begin{minted}[xleftmargin=20pt,linenos,bgcolor=bg]{python}
# losses
pixel_opt:  # pixel-wise 损失函数项,与模型有关
  type: L1Loss  # 设置为需要使用的损失函数类名
  ...
\end{minted}
\end{enumerate}

\begin{hl} % ---------------- Highlight block ---------------- %
    \textbf{添加非 \texttt{Class} 的损失函数}

    在实际使用情况中,我们会遇到一些损失函数,他们不是以类 (\texttt{Class} )的形式出现,而是普通的函数。比如 StyleGAN2 中使用的 \href{https://github.com/XPixelGroup/BasicSR/blob/master/basicsr/losses/gan_loss.py}{r1\_penalty} 和 \href{https://github.com/XPixelGroup/BasicSR/blob/master/basicsr/losses/gan_loss.py}{gradient\_penalty\_loss}。

    此时,我们不再以注册机制的方式使用,而是直接在模型中调用相关函数。
    \begin{minted}[xleftmargin=20pt,linenos,breaklines,bgcolor=bg]{python}
from basicsr.losses.gan_loss import r1_penalty

class StyleGAN2Model(BaseModel):
    ...
    def optimize_parameters(self, current_iter):
        ...
        real_pred = self.net_d(self.real_img)
        l_d_r1 = r1_penalty(real_pred, self.real_img)  # 直接调用损失函数 r1_penalty
        ...
\end{minted}
\end{hl}

% ------------------------------------------------------------------------------
\section{添加修改指标}\label{howto:add_metric}

\begin{enumerate}[第 1 步]
    \item 指标的存放与命名:指标文件存放在 \href{https://github.com/XPixelGroup/BasicSR/tree/master/basicsr/metrics}{basicsr/metrics/} 文件夹下。对于命名规则无要求,一般直接以功能命名即可,如 \href{https://github.com/XPixelGroup/BasicSR/tree/master/basicsr/metrics/psnr_ssim.py}{basicsr/metrics/psnr\_ssim.py}。

    \item 编写自定义指标:在指标文件中对自定义指标函数进行命名,\textbf{需要注意新建函数名不能与已有函数名重复,否则会导致后续注册机制报错}。关于指标的功能详解见第\ref{chaper:metrics}章:\nameref{chaper:metrics}。在编写完自定义指标后,注意在 \href{https://github.com/XPixelGroup/BasicSR/tree/master/basicsr/metrics/__init__.py}{basicsr/metrics/\_\_init\_\_.py} 文件中对添加的自定义指标进行导入。以 \textbf{calculate\_psnr} 为例:
          \begin{minted}[xleftmargin=20pt,linenos,bgcolor=bg]{python}
from .psnr_ssim import calculate_psnr
\end{minted}

    \item 注册指标:用户需要对新建的指标函数进行注册。注册机制的原理详见章节\ref{code_structure:register}:\nameref{code_structure:register}。此处具体操作为,首先对 \textbf{METRIC\_REGISTRY} 函数进行导入,然后在新建函数上方添加修饰器来注册新建函数。以 \href{https://github.com/XPixelGroup/BasicSR/tree/master/basicsr/metrics/psnr_ssim.py}{psnr\_ssim.py} 中的 \textbf{calculate\_psnr} 为例:
          \begin{minted}[xleftmargin=20pt,linenos,bgcolor=bg]{python}
from basicsr.utils.registry import METRIC_REGISTRY

@METRIC_REGISTRY.register()
def calculate_psnr(img, img2, ... ):
    ...
\end{minted}

    \item 在配置文件中设置自定义指标:将配置文件(即 YAML文件)中 \textbf{validation settings} 部分中 \textbf{metric} 部分中的  \textbf{type} 参数设置为新建的指标函数名即可。指标的其他参数设置对应其功能部分代码。以使用 \href{https://github.com/XPixelGroup/BasicSR/tree/master/basicsr/metrics/psnr_ssim.py}{psnr\_ssim.py} 中的 \textbf{calculate\_psnr} 为例:
          \begin{minted}[xleftmargin=20pt,linenos,bgcolor=bg]{python}
# validation settings
val:
  ...
  metrics:
    psnr:  # 指标名称,可以是任意的,用于标记
      type: calculate_psnr  # 设置为需要使用的指标函数名
      ...
\end{minted}
\end{enumerate}

\begin{note} % ---------------- Note block ---------------- %
    \textbf{注意}

    \begin{itemize}
        \item 指标的新建注册机制和其他几类 (dataset, model, arch, loss) 不同:1) 需要显式地在 \href{https://github.com/XPixelGroup/BasicSR/tree/master/basicsr/metrics/__init__.py}{basicsr/metrics/\_\_init\_\_.py} import 函数;2) 新建的是函数而不是类
        \item 目前指标主要是在 Numpy 上计算的,比如 \texttt{calculate\_psnr}、\texttt{calculate\_ssim},有些也提供了基于 PyTorch 计算的版本,比如 \texttt{calculate\_psnr\_pt}、\texttt{calculate\_psnr\_pt}
    \end{itemize}
\end{note}

\end{document}