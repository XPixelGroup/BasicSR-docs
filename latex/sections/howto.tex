\documentclass[../main.tex]{subfiles}

\begin{document}

\chapter{如何添加与修改}
\vspace{-2cm}

该部分主要介绍如何在BasicSR框架中添加自定义的 Data Loader ,网络结构( Architecture ),模型( Model ),损失函数( Loss )以及指标( Metric )。使用者需要关注四个方面,即相关文件的存放和命名,编写自定义文件,注册新添加类,以及在配置文件中进行设置。这一部分的内容大体上十分相似,使用者只要对某一个模块比较熟悉(如添加修改网络结构),即可快速类比至其他各个部分。在添加新的自定义模块时,理解并参考已有文件可以帮助使用者快速使用。

值得提及的是,当用户使用 \textbf{BasicSR-template} 进行开发,尤其是针对\textbf{损失函数}和\textbf{指标}模块时,以下操作可能并不完全适用,具体详见 \textbf{BasicSR-template} 相关部分。

\section{添加修改 Data Loader}
\begin{enumerate}
	\item Data Loader 文件的存放与命名:Data Loader文件存放在 \href{https://github.com/XPixelGroup/BasicSR/tree/master/basicsr/data}{basicsr/data/} 文件夹下。例如,\href{https://github.com/XPixelGroup/BasicSR/tree/master/basicsr/data}{basicsr/data/paired\underline{~}image\underline{~}dataset.py}。用户可根据需求对已有的 Data Loader 进行修改,或是添加自定义 Data Loader 文件。在创建新的自定义 Data Loader 文件时,注意文件名需以  \textbf{\underline{~}dataset.py} 作为结尾。
	
	\item 编写自定义 Data Loader :在 Data Loader 文件中对自定义 Data Loader 类进行命名,\textbf{需要注意新建类名不能与已有类名重复,否则会导致后续注册机制报错}。关于 Data Loader 文件中函数功能详解见\ref{Data Loader},此处不再赘述。对于需要手工设置的参数,用户可以灵活利用 \textbf{opt} 参数从配置文件中读取。
	
	\item 注册 Data Loader :用户需要对新建的 Data Loader 类进行注册。注册机制的原理详见\ref{Register}。此处具体操作为,首先对 \textbf{DATASET\underline{~}REGISTRY} 函数进行导入,然后在新建类上方调用注册函数。以 \href{https://github.com/XPixelGroup/BasicSR/tree/master/basicsr/data/paired_image_dataset.py}{paired\underline{~}image\underline{~}dataset.py} 中的 \textbf{PairedImageDataset} 为例:
\begin{minted}[xleftmargin=20pt,linenos,bgcolor=bg]{python}
from basicsr.utils.registry import DATASET_REGISTRY

@DATASET_REGISTRY.register()
class PairedImageDataset(data.Dataset):
    ...
\end{minted}

	\item 在配置文件中设置自定义 Data Loader :将配置文件(即YML文件)中 \textbf{datasets} 部分中 \textbf{type} 参数设置为新建的 Data Loader 类名即可。
	该部分其余参数的功能与 Data Loader 中用户自定义的功能对应。以使用 \href{https://github.com/XPixelGroup/BasicSR/tree/master/basicsr/data/paired_image_dataset.py}{paired\underline{~}image\underline{~}dataset.py} 中的 \textbf{PairedImageDataset} 为例:
\begin{minted}[xleftmargin=20pt,linenos,bgcolor=bg]{python}
# dataset and data loader settings
dataset:
  ...
  type: PairedImageDataset # 设置为需要使用的 Data Loader 类名
  ...
\end{minted}
\end{enumerate}

\section{添加修改网络结构}

\begin{enumerate}
	\item 网络结构文件的存放与命名:网络结构文件存放在 \href{https://github.com/XPixelGroup/BasicSR/tree/master/basicsr/archs}{basicsr/archs/} 文件夹下。例如,\href{https://github.com/XPixelGroup/BasicSR/tree/master/basicsr/archs}{basicsr/archs/srresnet\underline{~}arch.py}。用户可根据需求对已有的网络结构进行修改,或是添加自定义网络结构文件。在创建新的自定义网络结构文件时,注意文件名需以  \textbf{\underline{~}arch.py} 作为结尾。
	
	\item 编写自定义网络结构:在网络结构文件中对自定义网络结构类进行命名,\textbf{需要注意新建类名不能与已有类名重复,否则会导致后续注册机制报错}。关于网络结构文件中的函数功能详解见\ref{Architecture}。对于需要手工设置的参数,用户可以灵活利用 \textbf{opt} 参数从配置文件中读取。
	
	\item 注册网络结构:用户需要对新建的网络结构类进行注册。注册机制的原理详见\ref{Register}。此处具体操作为,首先对 \textbf{ARCH\underline{~}REGISTRY} 函数进行导入,然后在新建类上方调用注册函数。以 \href{https://github.com/XPixelGroup/BasicSR/tree/master/basicsr/archs/srresnet_arch.py}{srresnet\underline{~}arch.py} 中的 \textbf{MSRResNet} 为例:
\begin{minted}[xleftmargin=20pt,linenos,bgcolor=bg]{python}
from basicsr.utils.registry import ARCH_REGISTRY

@ARCH_REGISTRY.register()
class MSRResNet(nn.Module):
...
\end{minted}
	
	\item 在配置文件中设置自定义网络结构:将配置文件(即YML文件)中 \textbf{network structures} 部分中的 \textbf{type} 参数设置为新建的网络结构类名即可。
	该部分其余参数的功能与模型和网络结构中用户自定义的功能对应。以使用 \href{https://github.com/XPixelGroup/BasicSR/tree/master/basicsr/archs/srresnet_arch.py}{srresnet\underline{~}arch.py} 中的 \textbf{MSRResNet} 为例:
\begin{minted}[xleftmargin=20pt,linenos,bgcolor=bg]{python}
# network structures
network_g: # g网络设置
  ...
  type: MSRResNet # 设置为需要使用的网络结构类名
  ...
\end{minted}
\end{enumerate}

\section{添加修改模型}

\begin{enumerate}
	\item 模型文件的存放与命名:模型文件存放在 \href{https://github.com/XPixelGroup/BasicSR/tree/master/basicsr/models}{basicsr/models/} 文件夹下。例如,\href{https://github.com/XPixelGroup/BasicSR/tree/master/basicsr/models}{basicsr/archs/sr\underline{~}model.py}。用户可根据需求对已有的模型进行修改,或是添加自定义模型文件。在创建新的自定义模型文件时,注意文件名需以  \textbf{\underline{~}model.py} 作为结尾。
	
	\item 编写自定义模型:在模型文件中对自定义模型类进行命名,\textbf{需要注意新建类名不能与已有类名重复,否则会导致后续注册机制报错}。关于模型文件中的函数功能详解见\ref{Model}。模型部分涉及的函数较多,但一般情况下需要改写的部分非常有限。用户往往只需要继承已有模型,并对需要更改的函数进行重构即可。以 \href{https://github.com/XPixelGroup/BasicSR/tree/master/basicsr/models}{basicsr/archs/swinir\underline{~}model.py} 中的 \textbf{SwinIRModel} 为例,该模型相较于图像超分通用的 \href{https://github.com/XPixelGroup/BasicSR/tree/master/basicsr/models}{basicsr/archs/sr\underline{~}model.py} 中的 \textbf{SRModel} 仅需更改test函数,因此 \textbf{SwinIRModel} 类在继承了 \textbf{SRModel} 的基础上只对 \textbf{test} 函数进行了重构:
\begin{minted}[xleftmargin=20pt,linenos,bgcolor=bg]{python}
class SwinIRModel(SRModel): # SwinIRModel 继承自 SRModel
    def test(self): # 重构 test 函数
        ...
\end{minted}
	
	\item 注册模型:用户需要对新建的模型类进行注册。注册机制的原理详见\ref{Register}。此处具体操作为,首先对 \textbf{MODEL\underline{~}REGISTRY} 函数进行导入,然后在新建类上方调用注册函数。以 \href{https://github.com/XPixelGroup/BasicSR/tree/master/basicsr/models/sr_model.py}{sr\underline{~}model.py} 中的 \textbf{SRModel} 为例:
\begin{minted}[xleftmargin=20pt,linenos,bgcolor=bg]{python}
from basicsr.utils.registry import MODEL_REGISTRY

@MODEL_REGISTRY.register()
class SRModel(nn.Module):
    ...
\end{minted}
	
	\item 在配置文件中设置自定义模型:将配置文件(即YML文件)中 \textbf{general settings} 部分中的 \textbf{model\underline{~}type} 参数设置为新建的模型类名即可。以使用 \href{https://github.com/XPixelGroup/BasicSR/tree/master/basicsr/models/sr_model.py}{sr\underline{~}model.py} 中的 \textbf{SRModel} 为例:
\begin{minted}[xleftmargin=20pt,linenos,bgcolor=bg]{python}
# general settings
...
type: SRModel # 设置为需要使用的模型类名
...
\end{minted}
除此之外,模型与整个配置文件的内容都是息息相关的,涉及到数据的读取与处理、模型框架、训练优化和测试评估等几乎所有部分的设置组成,而非一个独立的部分。用户在修改配置文件的结构时,建议参考已有模板,对模型修改的部分在配置文件中做对应处理。

\end{enumerate}

\section{添加修改损失函数}

\begin{enumerate}
	\item 损失函数的存放与命名:损失函数文件存放在 \href{https://github.com/XPixelGroup/BasicSR/tree/master/basicsr/losses}{basicsr/losses/} 文件夹下。损失函数功能性较为独立且代码量往往不多,因此一般无需单独创建文件,而是直接写在 \href{https://github.com/XPixelGroup/BasicSR/tree/master/basicsr/losses/losses.py}{basicsr/losses/losses.py} 文件中,如 \textbf{L1Loss} 就直接写在该文件中。
	
	\item 编写自定义损失函数:在损失函数文件中对自定义损失函数类进行命名,\textbf{需要注意新建类名不能与已有类名重复,否则会导致后续注册机制报错}。关于损失函数的功能详解见\ref{Loss}。在编写完自定义损失函数后,注意在 \href{https://github.com/XPixelGroup/BasicSR/tree/master/basicsr/losses/__init__.py}{basicsr/losses/\underline{~~}init\underline{~~}.py} 文件中对添加的自定义函数进行导入。以 \textbf{L1Loss} 为例:
\begin{minted}[xleftmargin=20pt,linenos,bgcolor=bg]{python}
from .losses import L1Loss
\end{minted}
	
	\item 注册损失函数:用户需要对新建的损失函数类进行注册。注册机制的原理详见\ref{Register}。此处具体操作为,首先对 \textbf{LOSS\underline{~}REGISTRY} 函数进行导入,然后在新建类上方调用注册函数。以 \href{https://github.com/XPixelGroup/BasicSR/tree/master/basicsr/losses/losses.py}{losses.py} 中的 \textbf{L1Loss} 为例:
\begin{minted}[xleftmargin=20pt,linenos,bgcolor=bg]{python}
from basicsr.utils.registry import LOSS_REGISTRY

@LOSS_REGISTRY.register()
class L1Loss(nn.Module):
    ...
\end{minted}
	
	\item 在配置文件中设置自定义损失函数:将配置文件(即YML文件)中 \textbf{losses} 部分中相应损失函数项的 \textbf{type} 参数设置为新建的损失类名即可。需要注意损失函数项的存在与模型有关。以使用 \href{https://github.com/XPixelGroup/BasicSR/tree/master/basicsr/losses/losses.py}{losses.py} 中的 \textbf{L1Loss} 为例:
\begin{minted}[xleftmargin=20pt,linenos,bgcolor=bg]{python}
# losses
pixel_opt: # pixel-wise 损失函数项,与模型有关
  type: L1Loss # 设置为需要使用的损失函数类名
  ...
\end{minted}
\end{enumerate}

\section{添加修改指标}

\begin{enumerate}
	\item 指标的存放与命名:指标文件存放在 \href{https://github.com/XPixelGroup/BasicSR/tree/master/basicsr/metrics}{basicsr/metrics/} 文件夹下。对于命名规则无要求,一般直接以功能命名即可,如 \href{https://github.com/XPixelGroup/BasicSR/tree/master/basicsr/metrics/psnr_ssim.py}{basicsr/metrics/psnr\underline{~}ssim.py}。
	
	\item 编写自定义指标:在指标文件中对自定义指标进行命名,\textbf{需要注意新建类名不能与已有类名重复,否则会导致后续注册机制报错}。关于指标的功能详解见\ref{Metric}。在编写完自定义指标后,注意在 \href{https://github.com/XPixelGroup/BasicSR/tree/master/basicsr/metrics/__init__.py}{basicsr/metrics/\underline{~~}init\underline{~~}.py} 文件中对添加的自定义指标进行导入。以 \textbf{calculate\underline{~}psnr} 为例:
\begin{minted}[xleftmargin=20pt,linenos,bgcolor=bg]{python}
from .psnr_ssim import calculate_psnr
\end{minted}
	
	\item 注册指标:用户需要对新建的指标类进行注册。注册机制的原理详见\ref{Register}。此处具体操作为,首先对 \textbf{METRIC\underline{~}REGISTRY} 函数进行导入,然后在新建类上方调用注册函数。以 \href{https://github.com/XPixelGroup/BasicSR/tree/master/basicsr/metrics/psnr_ssim.py}{psnr\underline{~}ssim.py} 中的 \textbf{calculate\underline{~}psnr} 为例:
\begin{minted}[xleftmargin=20pt,linenos,bgcolor=bg]{python}
from basicsr.utils.registry import METRIC_REGISTRY

@METRIC_REGISTRY.register()
class calculate_psnr(nn.Module):
    ...
\end{minted}
	
	\item 在配置文件中设置自定义指标:将配置文件(即YML文件)中 \textbf{validation settings} 部分中 \textbf{metric} 部分中的  \textbf{type} 参数设置为新建的指标类名即可。指标的其他参数设置对应其功能部分代码。以使用 \href{https://github.com/XPixelGroup/BasicSR/tree/master/basicsr/metrics/psnr_ssim.py}{psnr\underline{~}ssim.py} 中的 \textbf{calculate\underline{~}psnr} 为例:
\begin{minted}[xleftmargin=20pt,linenos,bgcolor=bg]{python}
# validation settings
val:
  ...
  metrics:
    psnr: # 指标名称,可以是任意的
      type: calculate_psnr # 设置为需要使用的指标类名
      ...
\end{minted}
\end{enumerate}

\end{document}